\newpage
\subsection{Semmelknödel (2 Portionen)}
\paragraph{Zutaten}
\begin{itemize}[noitemsep]
	\item 6 Brötchen (altbacken)
	\item 3 EL Petersilie
	\item 10g Butter
	\item 1 Zwiebel
	\item 250ml Milch
	\item 3 Eier
	\item Salz und Pfeffer aus der Mühle
	\item ggf. Semmelbrösel zum Binden
	\item Salzwasser oder Brühe
\end{itemize}
\begin{enumerate}[noitemsep]
	\item die Brötchen in kleine Würfel oder dünne Scheiben schneiden
	\item Petersilie feinhacken und die Zwiebel kleinschneiden
	\item Petersilie und Zwiebeln in der Butter anschwitzen, aber keine Farbe nehme lassen
	\item dann mit den Bröchenwürfeln (in einer Schüssel?) mischen 
	\item die Milch bis kurz vorm Kochen erhitzen und über die Brotwürfel gießen
	\item dann ca. 10 Minuten quellen lassen 
	\item Eier mit Salz und Pfeffer verquirlen und dann dazugeben und alles zu einem Teig verrühren 
	\begin{description}[noitemsep, nolistsep]
		\item $\rightarrow$ Teig soll nicht zu fest sein
		\item $\rightarrow$ Ist der Teig zu fest: Semmelbrösel einrühren
	\end{description}
	\item mit nassen Händen tennisballgroße Knödel formen und in der siedenden (nicht sprudelnd kochenden) Flüssigkeit 20 Minuten gar ziehen lassen
	\begin{description}[noitemsep, nolistsep]
		\item fertig, sobald die Knödel nach oben steigen
	\end{description}
\end{enumerate}
\placegraphic{Bilder/Semmelknoedel}{0.85}