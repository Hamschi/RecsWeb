\subsection{Brötchen}
\paragraph{Zutaten}
\begin{itemize}[noitemsep]
	\item 500g Dinkelmehl Typ 630 
	\begin{description}[noitemsep, nolistsep]
		\item $\rightarrow$ alternativ Weizenmehl Typ 550
	\end{description}
	\item 300g Wasser (kalt) 
	\item 20g Olivenöl 
	\item 1 EL Salz
	\item 1g Hefe (frisch) 
	\begin{description}[noitemsep, nolistsep]
		\item $\rightarrow$ alternativ 0.5g Trockenhefe
	\end{description}
	\begin{description}[noitemsep, nolistsep]
		\item $\rightarrow$ entspricht einer Erbsengroßen Menge 
	\end{description}
	\item 1/2 Tasse Wasser (heiß)
\end{itemize}

\paragraph{Zubereitung}
\begin{enumerate}[noitemsep]
	\item Hefe im Wasser auflösen
	\item Olivenöl dazugeben
	\item Das Salz mit dem Mehl vermischen
	\item Das Mehl in die Flüssigkeit geben und durchmischen bis das ganze Mehl gebunden ist
	\item Den Teig bedecken und bei Zimmertemperatur über die Nacht lang ruhen lassen 
	\begin{description}[noitemsep, nolistsep]
		\item $\rightarrow$ Teig sollte sich ungefähr verdoppelt haben 
	\end{description}
	\item 1 Prise Mehl über den Teig und die Arbeitsoberfläche geben
	\item Teig auf die Arbeitsoberfläche geben und 9 Stücke portionieren (ca. 90g pro Brötchen)
	\item Den Rand einer Portionen immer und immer wieder in die Mitte ziehen und dann Brötchen formen
	\begin{description}[noitemsep, nolistsep]
		\item Vorgang für alle Portionen wiederholen 
	\end{description}
	\item Brötchen mit ausreichend Abstand auf ein Backblech mit Backpapier platzieren, abdecken und 30-45 Minuten lang gehen lassen 
	\item Brötchen bei 250\textdegree C Ober-Unterhitze vorheizen
	\item Währenddessen: eine Feuerfeste Schale in die unterste Ebene des Backofens legen 
	\item Brötchen mit einem Sägemesser oder einem Scharfen Messer einschneiden (ca. 2mm) 
	\item Brötchen mit Wasser besprühen
	\item Brötchen in den Backofen geben und die 1/2 Tasse Wasser in den Behälter geben
	\item 20 Minuten lang backen
	\item Brötchen auf einem Gitter auskühlen lassen
\end{enumerate}
\placeimage{Bilder/Broetchen}{11.5cm}{-4cm}{0.45}