\newpage
\titlelink{Lahmacun (9 Stück)}{https://youtu.be/W1FfuyaDYow?si=L6p3EQiP_1uM8CtH}
\paragraph{Zutaten}
\begin{itemize}[noitemsep]
	\item 500g Weizenmehl (Typ 550 oder 405)
	\item 260ml Wasser (lauwarm)
	\item 15g frische Hefe (alternativ 5g Trockenhefe)
	\vspace{0.5cm}
	\item 1 Spitzpaprika (rot)
	\item 500g Hackfleisch
	\item 2 Zwiebeln (klein)
	\item 2 Zehen Knoblauch
	\item 2 Tomaten
	\item 1 Handvoll Blattpetersilie (Blätter)
	\item 2 Peperoni (grün, frisch)
	\item 50ml Sonnenblumenöl
	\item Salz
	\item Pfeffer
	\item 1 TL Paprikapulver
	\item 1 TL Pul Biber
	\item 1 EL Tomatenmark
	\item Zum Belegen: Zwiebelscheiben (optional mit Sumach), Blattpetersilie, Tomaten, Zitronensaft
\end{itemize}

\paragraph{Zubereitung}
\begin{enumerate}[noitemsep]
	\item Hefe im Wasser auflösen lassen
	\item Das Hefewasser und das Salz zum Mehl geben, verrühren und zu einem glatten Teig verarbeiten 
	\begin{description}[noitemsep, nolistsep]
		\item \rarrow Teig zum Kneten in die Länge ziehen 
	\end{description}
	\item Kleine Schüssel mit Öl ausreiben und den Teig reinlegen und mit Frischhaltefolie abgedeckt ruhen lassen, sodass sich der Teig verdoppelt (ca. 1h)
	\item Haut von den Tomaten abziehen, indem sie mit kochendem Wasser kurz übergossen und nach 10s abgeschreckt werden
	\item Tomaten klein schneiden, pürieren und in eine Schüssel geben
	\item Petersilie und Zwiebeln klein schneiden, pürieren und in die Schüssel geben
	\item Knoblauch und die Paprika (ohne Kerne) kleinschneiden, pürieren und in die Schüssel geben
	\item Peperoni kleinschneiden (Kerne optional), pürieren und in die Schüssel geben
	\item Paprikapulver, Pul Biber, Salz, Pfeffer, Öl und das Tomatenmark in einer Schüssel vermengen
	\item Püree zu den Gewürzen geben und alles gut vermengen
	\item Hackfleisch nun mit der Püreemasse vermengen
	\item Hackfleisch mit Frischhaltefolie abdecken (berührt das Fleisch direkt) und in den Kühlschrank legen bis es später verarbeitet wird
	\item Teig in 80g-Portionen unterteilen
	\item Teige in Kugeln formen, indem man den Teig wie Brötchen \glqq{}schleift\grqq{}/knetet (siehe Video) 
	\item Teig mit Hilfe eines Nudelholz dünn platt rollen
	\item Füllung bis zum Rand verteilen (ca. 0.5-1cm Rand kann übrig gelassen werden)
	\item Lahmacun in eine heiße Pfanne geben und einen Deckel drauf legen bis der Teig knusprig wird und leicht geröstet
\end{enumerate}
\placegraphic{Bilder/Lahmacun}{0.9}