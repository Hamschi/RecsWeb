\newpage
\subsection{Papparoti/Rotiboy (mexikanische Kaffeebrötchen)}
\paragraph{Zutaten}
\begin{itemize}[noitemsep]
	\item 110ml Milch
	\item 25g Zucker
	\item 4g Trockenhefe
	\item 1/2 Ei
	\item 200g Mehl
	\item 1/3 TL Salz
	\item 30g Butter
	\item 3g gesalzene Butter (oder laughing cow cheese)
	\vspace{0.5cm}
	\item 3g Instantkaffee
	\item 5ml heißes Wasser
	\item 1/2 Ei
	\item 35g Puderzucker
	\item 40g Mehl
	\item 40g Butter (Zimmertemperatur)
\end{itemize}
\paragraph{Zubereitung}
\begin{enumerate}[noitemsep]
	\item Zucker und Hefe zur Milch geben und verrühren bis diese sich darin auflösen
	\item Ei schlagen (damit man das Ei besser halbieren kann)
	\item Halbes Ei, Mehl und Salz in die Mischung geben und dann umrühren
	\item Danach: Teig innerhalb der Schüssel formen
	\item Danach: Teig auf die Bearbeitungsoberfläche legen und 10 Minuten lang durchkneten
	\begin{description}[noitemsep, nolistsep]
		\item $\rightarrow$ bis es weich wie Babyhaut ist und zugleich dünn sein kann
	\end{description}
	\item Danach: Teig in eine Kugel formen und plattdrücken
	\item Butter auf den Teig legen und weitere 10 Minuten durchkneten
	\item Teig in eine Kugel formen
	\item Teig eine Stunde lang ruhen lassen
	\item 1-2 Minuten lang kneten, um Blasen zu bekämpfen
	\item Teig in eine Kugel formen und in 6 gleichgroße Stücke unterteilen
	\item Teig zu Kugeln formen und 15 Minuten lang ruhen lassen
	\item Teig auf die Bearbeitungsfläche drücken
	\item Butter/ Käse von dem Teig umschlingen lassen
	\item 30-40 Minuten warten bis der Teig zu doppelter Größe heranwächst
	\item Instant Kaffee mit 5ml heißem Wasser mischen und rühren
	\item das halbe Ei, Puderzucker und Mehl dazugeben und alles gründlich verrühren
	\item Die Butter in der Pampe zerdrücken und mischen
	\item Füllung in eine Spritztüte geben und vom Zentrum aus beginnend bis zur Hälfte spiralförmig die Mischung spritzen
	\item 180\textdegree C Ober-/Unterhitze 18 Minuten lang backen 
\end{enumerate}
\placeimage{Bilder/Paparoti}{12.5cm}{-5cm}{0.5}