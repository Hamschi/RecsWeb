\newpage
\subsection{Knoblauchbrot}
\paragraph{Zutaten}
\begin{itemize}[noitemsep]
	\item 1 Brot oder was ähnliches (er empfiehlt Shabata)
	\item 220g Butter
	\item 4-6 Zehen Knoblauch (gehackt)
	\item 3-4 EL Petersilie (gehackt)
	\item 2 EL Parmesan 
	\item 1 TL Salz
\end{itemize}
\paragraph{Zubereitung}
\begin{enumerate}[noitemsep]
	\item die Butter in einer Schüssel weich streichen
	\item den Knoblauch und die Petersilie in die Schüssel geben
	\item den Parmesan in die Schüssel reiben und alles gut vermischen 
	\item das Brot längst halbieren und beide Seiten mit der Knoblauchbutter bestreichen
	\item Knoblauchbrot auf ein mit Alufolie belegtes Blech legen und für 10-12 Minuten im Ofen backen lassen (205\textdegree C) 
	\begin{description}[noitemsep, nolistsep]
		\item $\rightarrow$ fertig, sobald das Brot gold-braun ist und die Butter vom Brot aufgesogen wurde  
	\end{description}
	\item Danach: das Brot in Slices schneiden 
	\item das Brot noch mit Salz bestreuen und fertig
\end{enumerate}
\begin{figure}[h]
\centering
\includegraphics[width=1\textwidth]{Bilder/Knoblauchbrot}
\end{figure}