\newpage
\subsection{Chicken Burrito (5 Portionen)}
\paragraph{Zutaten}
\begin{itemize}[noitemsep]
	\item 600g Hähnchenbrust
	\item 1 TL Knoblauchpulver
	\item 1 TL Oregano (getrocknet)
	\item 1 TL Salz
	\item 2 TL Kreuzkümmel
	\item 2 TL Paprikapulver
	\item 1 Prise Pfeffer
	\item 3/4 TL Cayenne Pfeffer
	\item 2 EL Olivenöl
	\item 1/2 Zwiebel (gehackt)
	\item 2 Zehen Knoblauch (gehackt)
	\item 1 Paprika (rot) (gehackt)
	\item 400g Refried Beans
	\item 60 ml Wasser
	\vspace{0.5cm}
	\item 250g Cups roten mexikanischen Reis (alternativ: weißer Reis)
	\item 400g Mais 
	\item 150g geriebenen Cheddar (oder halt was anderes noch dazu)
	\item 50g Koriander
\end{itemize}
\paragraph{Zubereitung}
\begin{enumerate}[noitemsep]
	\item Öl gleichmäßig auf das Fleisch verteilen
	\item Fleisch nun in den Gewürzen vermengen 
	\item Fleisch anbraten, bis es dunkle Stellen aufweist
	\item Fleisch 2 Minuten kühlen lassen und anschließend in kleine Würfel schneiden
	\item Zwiebeln, Knoblauch und Paprika anbraten 
	\item Danach: die Refried Beans dazugeben und das Wasser reinschütten
	\item Nachdem sich die Bohnen \glqq{}auflösten\grqq{}, Fleisch dazugeben und alles umrühren
	\item das alles nun solange kochen, bis eine dickflüssige, saftige Masse entsteht
	\item Danach: 5 Minuten kühlen lassen
	\item Tortillas in der Mikrowelle warmmachen
	\item Reis im unteren Drittel verteilen
	\item Hähnchen darüber legen
	\item mit 1 EL Mais und eine \glqq{}Prise\grqq{} Koriander und einem EL Käse bestreuen 
	\item Burrito zusammenrollen, in Alufolie einwickeln und dann im Ofen für 10 Minuten bei 180\textdegree C backen oder alle Seiten (mit Folie) je 2 Minuten in einer trockenen Pfanne rösten
	\item Mögliche Dips: Avocado Sauce, Guacamole, Sour Cream, Salsa, Quesok, Nacho Cheese Dip...
\end{enumerate}
\placeimage{Bilder/ChickenBurrito}{11cm}{-4.7cm}{0.65}