\newpage
\subsection{Bruschetta}
\paragraph{Zutaten}
\begin{itemize}[noitemsep]
	\item Sauerteigbrot, Pane di Casa,... in Scheiben 
	\item frischer Basilikum
	\item Cherrytomaten oder Salattomaten
	\item 2 Zehen Knoblauch
	\item 1 rote Zwiebel (optional)
	\item Olivenöl
	\item Salz und Pfeffer
	\item Mozzarella (optional)
\end{itemize}
\paragraph{Zubereitung}
\begin{enumerate}[noitemsep]
	\item den "Kopf" der Tomate über einer Schüssel ausschneiden und dann die "Hülle" in kleine Stücke schneiden
	\begin{description}[noitemsep, nolistsep]
		\item $\rightarrow$ bzw. Salattomaten kleinschneiden
	\end{description}
	\item Olivenöl großzügig und gleichmäßig über die Tomaten schütten 
	\item mit Salt und Pfeffer abschmecken
	\item 1 Zehe Knoblauch feinschneiden und zu den Tomaten geben (ggf. gehackte Zwiebeln hinzugeben)
	\item Basilikumblätter kleinrupfen und hinzugeben und alles miteinander vermengen
	\item Mischung 30 Minuten einwirken lassen
	\item Möglichkeiten, das Brot zuzubereiten:
	\begin{description}[noitemsep, nolistsep]
		\item $\rightarrow$ Bratpfanne: auf hoher Hitze die Scheiben beidseitig anbraten (ohne Öl), bis sie schwarze Stellen aufweisen
		\item $\rightarrow$ Toaster: ist halt eine schnelle Alternative
		\item $\rightarrow$ Ofen: Scheiben auf Backpapier in den Ofen geben 
	\end{description}
	\item Knoblauchzehe über die Brotscheiben reiben 
	\item den Saft der Tomaten über die Brotscheiben verteilen und dann den Tomatenmix gleichmäßig auf den Brotscheiben verteilen 
	\item Optional: Extrasaft danach nochmal drüber 
	\item Alternativ: bisschen Olivenöl über die Brotscheiben verteilen, die halbierten Tomaten über das Brot reiben und dann dort behalten  und mit Basilikum belegen
\end{enumerate}
\placeimage{Bilder/Bruschetta}{11.5cm}{-4.5cm}{0.65}