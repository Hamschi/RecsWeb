\newpage
\subsection{Einfach nur Brot (not done yet)}
\paragraph{Zutaten}
\begin{itemize}[noitemsep]
	\item 250g Dinkelmehl Typ 630 (alternativ Weizenmehl Typ 550)
	\item 250g Dinkelvollkornmehl (Oder Weizenvollkornmehl)
	\item 150g Wasser (lauwarm)
	\item 200g Buttermilch (aus dem Kühlschrank)
	\item 1 EL Salz (10g)
	\item 5g Hefe (frisch)
	\vspace{0.5cm}
	\item 1 TL Koriandersamen
	\item 1 TL Fenchelsamen
	\item 1/2 TL Kümmel
\end{itemize}
\paragraph{Zubereitung}
\begin{enumerate}[noitemsep]
	\item Die Mehlsorten und das Salz miteinander vermischen
	\item Die Koriandersamen, Fenchelsamen und den Kümmel in einem Mörser zu einem Brotgewürz zerkleinern und davon 2 TL zur Mehlmischung geben und vermischen 
	\item Buttermilch zum Wasser geben und in der Flüssigkeit die Hefe auslösen lassen
	\item Die Flüssigkeit zu der Mehlmischung geben und alles zu einem Teig vermischen
	\item Teig auf dem Tisch kneten bis man einen schönen Teig hat (ca. 5 Minuten)
	\item Teig in die Schüssel geben, mit Frischhaltefolie bedecken und für 1h bei Zimmertemperatur ruhen lassen
	\item Danach: Teig aus der Schüssel entnehmen und die Seiten des Teiges in die Mitte ziehen 
	\item Teig in einen Gärkorb legen
	\begin{description}[noitemsep, nolistsep]
		\item $\rightarrow$ Alternativ eine Schüssel, die mit einem Leinentuch ausgelegt ist 
	\end{description}
	\item Das Tuch und den Teig mit Mehl bestreuen und das Brot mit der platten Seite auf das Tuch legen, mit dem Tuch zu wickeln und über Nacht im Kühlschrank stehen lassen
	\begin{description}[noitemsep, nolistsep]
		\item $\rightarrow$ 6\textdegree C sind ideal, 4-5\textdegree C etwas kalt
	\end{description}
	\item Den Gärkorb mit Frischhaltefolie beziehen, damit der Teig nicht austrocknet
	\item Danach: Brot aus dem Kühlschrank stellen und den Ofen auf 250 \textdegree C Ober-Unterhitze einstellen 
	\item Brot in einen Gusstopf legen
	\begin{description}[noitemsep, nolistsep]
		\item $\rightarrow$ Edelstahltopf geht auch, aber dieser sollte Deckelhitzebeständig sein
		\item $\rightarrow$ Backblech ist auch okay, aber nicht optimal 
	\end{description}
	\item Backpapier zu einem Quadrat falten und 1/4 wegschneiden (nicht an der Kante)
	\item Brot mit Mehl bestreuen und auf das Backpapier stürzen
	\item Brot mit einer Rasierklinge einschneiden (mittig sowie links rechts) 
	\item Brot in den Topf legen, Deckel drauf tun und Brot für 30 Minuten backen
	\item Danach: Backpapier entnehmen und Brot für 20 weitere Minuten bei 200-220\textdegree C backen
	\begin{description}[noitemsep, nolistsep]
		\item $\rightarrow$ Wasser in einen Behälter auf die unterste Ebene stellen
	\end{description}
\end{enumerate}


% 1 TL Koriander Samen 1 TL Fenchelsamen 1/2 TL Kümmel = 2 TL 