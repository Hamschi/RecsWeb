\clearpage
\titlelink{Acili Ezme (not done yet)}{https://www.instagram.com/reel/DIjuqy2ODeX/?igsh=MWJocG83cjBzdXR6eg==}

\paragraph{Zutaten}
\begin{itemize}[noitemsep]
	\item 2 Tomaten
	\item 1 Zwiebel
	\item 2 Spitzpaprika (grün)
	\item 2 Spitzpaprika (rot)
	\item 1-2 Knoblauchzehen 
	\item 1-2 Chilischoten
	\item 1 Handvoll Petersilie
	\item 1 EL Paprikamark
	\item 1 EL Tomatenmark
	\item 1 EL Sumak
	\item 1/2 Zitrone 
	\item 3 EL Olivenöl
	\item 3 EL Granatapfelsirup
	\item Salz
	\item Pfeffer
\end{itemize}

\paragraph{Zubereitung}
\begin{enumerate}[noitemsep]
	\item Tomaten, Paprika Zwiebel, Knoblauch, Chili und Petersilie grob schneiden
	\item Alle gehackten Zutaten in einen Mixer geben und mixen, bis es stückig klein ist
	\item Masse in einen Sieb geben und die Flüssigkeit ausdrücken
	\item Paprikamark, Tomatenmark, Saft der Zitrone, Olivenöl, Granatapfelsirup, Salz und Pfeffer in einer Schüssel vermengen
	\item Gemischte Masse hinzugeben und zu einer Masse vermengen
	\item Vor dem Servieren für mindestens 1h im Kühlschrank ziehen zu lassen
\end{enumerate}