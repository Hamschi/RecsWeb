\newpage
\subsection{Tomatensoße (ca. 500ml)}
\paragraph{Zutaten}
\begin{itemize}[noitemsep]
	\item 1l Wasser
	\item 1kg Tomaten
	\item 1 Zwiebel
	\item 2 EL Olivenöl
	\item 1 TL Salz
	\item 1 TL Salz
	\item 1 TL Zucker
	\item Basilikumblätter
\end{itemize}
\paragraph{Zubereitung}
\begin{enumerate}[noitemsep]
	\item Wasser in einem Wasserkocher erhitzen 
	\item Währenddessen: Tomaten waschen und in einen großen Topf legen 
	\item das kochendheiße Wasser zu den Tomaten geben und 5 Minuten einwirken lassen
	\item Zwiebel schälen und kleinschneiden
	\item die Zwiebel dann mit dem Olivenöl in einem Topf goldbraun anbraten
	\item Nach den 5 Minuten: Wasser ablassen und die Tomaten schälen 
	\begin{description}[noitemsep, nolistsep]
		\item $\rightarrow$ Tipp: ein Kreuz auf die Oberfläche der Tomate einzuschneiden hilft die Schale zu entfernen
	\end{description}
	\item Tomaten in zwei Hälften schneiden und den weißen Kern der Tomate entnehmen
	\item Tomaten in längliche Streifen schneiden
	\item Sobald Zwiebeln gold sind: Tomaten zu den Zwiebeln schütten und mit Salz und Zucker abschmecken
	\item die Tomaten 25-30 Minuten bei mittlerer/hoher Hitze ohne Deckel kochen lassen
	\item Danach: ggf. mit einem Mixer pürieren
	\item Basilikum in kleine Stücke schneiden und zu Soße hinzugeben
\end{enumerate}
\begin{figure}[h]
\centering
\includegraphics[width=.77\textwidth]{Bilder/Tomatensosse}
\end{figure}