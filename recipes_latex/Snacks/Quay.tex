\selectlanguage{vietnamese}
\subsection{Quẩy (Vietnamese fried breadsticks)}
\selectlanguage{ngerman}
\paragraph{Zutaten}
\begin{itemize}[noitemsep]
	\item 750g Mehl
	\item 1,5 EL Trockenhefe
	\item 2 TL Salz
	item 400ml Milch (warm)
	\item bisschen
\end{itemize}
\paragraph{Zubereitung}
\begin{enumerate}[noitemsep]
	\item In einem Behälter die Trockenhefe mit dem Wasser mischen
	\item dort nun Salz hinzufügen
	\item Verrühren bis sich die Hefe im Wasser aufgelöst hat
	\item Mehl in eine andere große Schüssel schütten
	\item die Mischung ins Mehl hinzugeben
	\item das Zeug zu einem Teig verkneten 
	\begin{description}[noitemsep, nolistsep]
		\item $\rightarrow$ Teig muss klebrig sein und nicht so Kaugummi-fest sein
		\item $\rightarrow$ Wenn man den Teig zieht, sollte dieser so elastisch sein wie Pizzakäse
	\end{description}
	\item Teig ruhen lassen
	\item Danach: ein wenig Mehl auf den Teig streuen
	\item den gesamten Teig zu einer riesigen Kugel formen 
	\item Mehl auf das Schneidebrett verteilen
	\item eine Handvoll Teig nehmen und in einen Rechteck formen
	\begin{description}[noitemsep, nolistsep]
		\item $\rightarrow$ Teig soll so hoch sein wie Apfelringe dick sind
	\end{description}
	\item zwei dieser dünnen Streifen nehmen und in sich zusammen 2-3 drehen, sodass sie aussehen wie das Pirulo Tropical Eis 
	\begin{description}[noitemsep, nolistsep]
		\item $\rightarrow$ nicht zu feste, sonst kann sich der Teig nicht gut ausdehnen
		\item $\rightarrow$ alternativ: zwei Streifen nebeneinander legen
	\end{description}
	\item Teigpaare frittieren bis sie goldbraun sind und ab und zu wenden (6-8 Minuten)
	\begin{description}[noitemsep, nolistsep]
		\item $\rightarrow$ Platz lassen, damit sich der Teig noch ausdehnen kann
	\end{description}
	\item Währenddessen: Teigpaare schneiden, um Zeit zu sparen
\end{enumerate}
\placeimageRotated{Bilder/Quay}{14.5cm}{-5cm}{0.45}