\newpage
\titlelink{Mikrowellenchips}{https://youtu.be/v514hwoC_fY?si=KxbEVOGPHW5EF2uw}
\paragraph{Zutaten}
\begin{itemize}[noitemsep]
	\item Backpapier
	\item Kartoffel (mehlig kochend, groß)
	\item Rapsöl
	\item Salz
\end{itemize}
\paragraph{Zubereitung}
\begin{enumerate}[noitemsep]
	\item Backpapier 2x (halbiert) falten
	\item Ecke zur anderen Ecke falten 
	\item Und dann iwie falten, dass es die Form eines Papierfliegers einnimmt
	\item Schneide von der Spitze so viel ab, dass es dem Radius des benutzten Tellers entspricht
	\item Kartoffeln waschen und anschließend mit einem Gemüsehobel in dünne Scheiben schneiden
	\item Kartoffeln in kaltes Wasser legen und möglichst viel Stärke abwaschen
	\item Kartoffeln mit einem Küchentuch trocknen
	\item Pinsel etwas Öl über das Backpapier und lege das Backpapier auf den Teller
	\item Platziere so viele Kartoffelscheiben wie möglich auf den Teller (keine Überlappungen)
	\item In der Mikrowelle für 2.5-3 Minuten (800 Watt) backen 
	\item Chips wenden und für weitere 4 Minuten backen 
	\begin{description}[noitemsep, nolistsep]
		\item \rarrow regelmäßig prüfen, dass die nicht verbrennen
		\item \rarrow Stelle sollten anfangen braun zu werden
		\item \rarrow Für mich haben 3 Minuten ca. gereicht, damit sie die gewünschte Konsistenz erreichten (ohne zu wenden)
	\end{description}
	\item Wenn noch Ränder weich sind: für paar Sekunden bei 400 Watt in die Mikrowelle
	\item Transferiere die Chips auf einen kalten Teller und bestreue sie mit Salz
	\item Tipp \#1: Die Chips werden erst so wirklich knusprig, wenn sie abkühlen
\end{enumerate}
\placegraphic{Bilder/Mikrowellenchips}{0.6}