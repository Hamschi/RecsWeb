\clearpage
\titlelink{Kabayaki Aubergine (2 Portionen) (not done yet)}{https://youtu.be/NiKQ5n4F9KA?si=XJ5ckIcTS2uMpZB4}

\paragraph{Zutaten}
\begin{itemize}[noitemsep]
	\item 3 Auberginen
	\item 1 EL Öl
	\item 1/2 EL Mehl
	\item 1 EL Muscovado Zucker (unraffinierter Rohrzucker)
	\item 2 EL Sojasoße (dunkel)
	\item 2 EL Sake
	\item 1.5 EL Mirin
	\item Kizami Nori (geraspeltes Nori)
	\item Frühlingszwiebeln (fein geschnitten)
	\item Sesam
\end{itemize}

\paragraph{Zubereitung}
\begin{enumerate}[noitemsep]
	\item Auberginen schälen und in eine hitzebeständige Schüssel legen
	\item Schüssel mit Frischhaltefolie bedecken und für 3 Minuten bei 600W in die Mikrowelle stellen
	\item Zucker, Sojasoße, Sake und Mirin in einer Schüssel verrühren
	\item Sobald die Auberginen kühl genug sind: längst anschneiden ohne komplett durchzuschneiden
	\item Öffne die Auberginen wie ein Buch
	\begin{description}[noitemsep, nolistsep]
		\item \rarrow mehrere Einstiche können beim Öffnen helfen
	\end{description}
	\item Mit der Oberfläche des Messers auf die Oberfläche drücken, um sie weiter zu plätten
	\item Die offene Seite der Aubergine mit Mehl bestäuben
	\item Öl in einer Pfanne erhitzen und Auberginen mit der Mehlseite nach unten in die Pfanne legen
	 \item Sobald gold-braun: Die Aubergine flippen
	 \item Soße in die Pfanne dazugeben und die Auberginen vorsichtig in der Pfanne bewegen, damit sie gleichmäßig von der Soße bedeckt sind
	 \item Sobald die Soße dick ist: Pfanne vom Herd nehmen
	 \item Kizami Nori auf den Schüsseln Reis verteilen und anschließend die Aubergine darüber legen
	 \item Mit Frühlingszwiebeln und Sesam verzieren
\end{enumerate}