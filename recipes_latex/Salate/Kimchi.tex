\newpage
\subsection{Kimchi}
\paragraph{Zutaten}
\begin{itemize}[noitemsep]
	\item 2,7kg Chinakohl (ca. 3 Stück)
	\item Salz
	\vspace{0.5cm}
	\item 2 Cups Wasser
	\item 2 EL Klebreismehl 
	\item 2 EL Zucker (braun)
	\vspace{0.5cm}
	\item 1/3 Rettich (2 Cups)
	\item 1-2 Karotten (1 Cup)
	\item 7-8 Stücke Frühlingszwiebeln (grüner Teil)
	\item 1 Cup Schnittlauch (koreanisch) (alternativ: 3 grüne Zwiebeln, weißer Teil)
	\item 1 Cup Minari (optional)
	\vspace{0.5cm}
	\item 24 Zehen Knoblauch
	\item 2 TL Ingwer
	\item 1 Zwiebel (mittelgroß)
	\item 120ml Fischsoße
	\item 1/4 Cup Saeujeot (inkl. salziger Lake)
	\item 2 Cups gochugaru (Red Pepper Flakes)
\end{itemize}

\paragraph{Zubereitung}
\begin{enumerate}[noitemsep]
	\item Wenn die Kohlenden zu sehr hervorstehen: schneide sie mit einem Messer ab
	\item Schneide einen kurzen Schlitz an der Basis des Kohls ein
	\item Ziehe die Hälften vorsichtig auseinander, sodass der Kohl aufgespalten wird
	\item Schneide einen Schlitz durch den Kern jeder Hälfte (ca. 5 Zentimeter)
	\begin{description}[noitemsep, nolistsep]
		\item $\rightarrow$ Kohlblätter sollen locker sein, aber immer noch am Kern befestigt bleiben
	\end{description}
	\item Tauche die Kohlhälften in einen großen Wasserbehälter, um sie feucht zu machen
	\item hebe jedes Blatt an und gebe Salz dazwischen
	\begin{description}[noitemsep, nolistsep]
		\item $\rightarrow$ mehr Salz an den Stielen, wo die Blätter dicker sind
	\end{description}
	\item Das Salz 2 Stunden einwirken lassen
	\item Danach: Kohlhälften mehrmals unter kaltem, fließendem Wasser waschen, um Salz und Schmutz zu entfernen
	\begin{description}[noitemsep, nolistsep]
		\item $\rightarrow$ Währenddessen: Teile die Hälften entlang der zuvor eingeschnittenen Schlitze in Viertel auf
	\end{description}
	\item Schneide die Kerne ab, spüle den Kohl noch einmal ab und lege den in ein Sieb über eine Schüssel, um den abtropfen zu lassen
	\vspace{0.5cm}
	\item In einem kleinen Topf Klebreis zum Mehl geben und auf mittlerer Hitze umrühren bis Blasen zu erkennen sind (ca. 10 Minuten)
	\item Füge den Zucker hinzu und verrühren alles für eine weitere Minute
	\item Danach: Topf vom Herd nehmen und alles komplett abkühlen lassen
	\item Rettich, Karotten in Streifen schneiden, Frühlingszwiebeln in dünne Rohre, Knoblauch fein
	\item Gebe den Brei in eine große Rührschüssel
	\item Füge Knoblauch, Ingwer, Zwiebel Fischsoße, Saeujeot und gochugaru hinzu und vermenge alles gut zusammen bis die Mischung zu einer dünnen Pasten wird
	\item Füge den Rettich, die Karotte und die Frühlingszwiebeln (und das Optionale Zeug) hinzu und vermenge alles gut
	\item Verteile in einer großen Schüssel etwas Kimchi-Paste auf jedes Kohlblatt
	\item Wenn jedes Blatt zu einem Viertel mit Paste bedeckt ist, wickele das Blatt um sich selbst und lege es in ein Gefäß
	\item Entweder direkt essen oder lasse es paar Tage fermentieren
	\item Sobald es anfängt zu gären, lagere das Kimchi im Kühlschrank 
	\begin{description}[noitemsep, nolistsep]
		\item $\rightarrow$ sobald es anfängt zu gären, wird es sauer riechen und schmecken
		\item $\rightarrow$ Wenn man das Kimchi mit einem Löffel auf der Oberseite drückt, werden Bläschen von unten freigesetzt
	\end{description}
\end{enumerate}
\placegraphic{Bilder/Kimchi}{0.8}