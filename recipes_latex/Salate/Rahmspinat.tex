\newpage
\titlelink{Rahmspinat (3-4 Portionen)}{https://youtu.be/vbhtBPLlgUA?si=yIP44M-r0w1DAz78}
\paragraph{Zutaten}
\begin{itemize}[noitemsep]
	\item 400g Blattspinat
	\item 1-2 Schalotten
	\item 1 Zehe Knoblauch
	\item Butter zum Braten
	\item 500ml Sahne
	\item Salz
	\item Pfeffer
	\item Muskatnuss
\end{itemize}


\paragraph{Zubereitung}
\begin{enumerate}[noitemsep]
	\item Eiswasser vorbereiten zum Abschrecken des Spinats 
	\item Wasser in einem Topf erhitzen lassen und den Spinat kurz abkochen
	\item Sobald der Spinat zusammengefallen ist: Spinat gut abtropfen lassen und in das Eiswasser geben
	\item Spinat abgießen, in ein Küchentuch legen und darin das Wasser ordentlich auspressen
	\item Schalotten und Knoblauch in kleine Würfel schneiden
	\item In der Butter die Schalotten und den Knoblauch anschwitzen
	\item Sobald die Schalotten und der Knoblauch weich sind: Sahne hinzugeben
	\item Salz, Pfeffer und Muskatnuss hinzugeben
	\item Sobald die Sahne kocht: Hitze reduzieren und die Sahne einköcheln lassen bis sie leicht dick wird
	\item Währenddessen: Spinat klein hacken
	\item Sobald die Sahne leicht dick ist: Spinat hinzugeben und in der Sahne vermengen
\end{enumerate}

\placegraphic{Bilder/Rahmspinat}{0.7}