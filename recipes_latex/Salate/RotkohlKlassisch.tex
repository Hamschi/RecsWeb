\newpage
\subsection{Rotkohl (klassisch) (4 Portionen)}
\paragraph{Zutaten}
\begin{itemize}[noitemsep]
	\item 1 kg Rotkohl
	\item 50g Schweineschmalz (alternativ Butter)
	\item 1 Zwiebel
	\item 2 Apfel (sauer)
	\item 1 EL Zucker
	\item 1 Schuss Essig
	\item 150ml Wasser
	\item 100ml Rotwein (trocken)
	\item 1 Prise Salz
	\item 1 Lorbeerblatt
	\item 2 Nelke
	\item 1 TL Mehl 
	\item 1 Prise Pfeffer
\end{itemize}
\paragraph{Zubereitung}
\begin{enumerate}[noitemsep]
	\item Zuerst den Rotkohl putzen, waschen, vierteln und den Strunk herausschneiden
	\item Anschließend fein schneiden oder mit einem Küchenhobel fein hobeln
	\item Danach: die Zwiebel schälen und fein hacken
	\item Die Äpfel waschen, vierteln, entkernen und in feine Scheiben schneiden
	\item Das Schweineschmalz in einem Schmortopf erhitzen und die Zwiebel, den Zucker und die Äpfel darin andünsten
	\item Anschließend den Rotkohl zugeben, den Schuss Essig angießen und gut durchrühren
	\item Danach etwa 10 Minuten bei mittlerer Temperatur dünsten
	\item Nun den Rotkohl mit dem Wasser und dem Wein aufgießen, salzen und die Nelken sowie das Lorbeerblatt einlegen
	\item Dann zudecken und etwa 40 Minuten weiterdünsten, bis der Kohl weich ist
	\item Zuletzt den Rotkohl noch mit ein wenig heißem Wasser aufgießen und mit etwas Mehl binden
	\item Mit Rotwein, Salz und Pfeffer abschmecken und servieren
\end{enumerate}
\placeimage{Bilder/Rotkohl_klassisch}{12cm}{-4.8cm}{0.55}