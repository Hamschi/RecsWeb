\clearpage
\titlelink{Tarte Tatin (not done yet)}{https://youtu.be/NpJUTALL64I?si=N4q9OX_6Klr7JF6J}
\paragraph{Zutaten}
\begin{itemize}[noitemsep]
	\item 1 Rolle fertigen Blätterteig (Alternativ: Mürbeteig)
	\item 7-9 Äpfel (Braeburn, Coy Orange, Granny Smith,...)
	\item 120g Zucker
	\item 80g Butter
\end{itemize}

\paragraph{Zubereitung}
\begin{enumerate}[noitemsep]
	\item Blätterteig rund ausschneiden (ca. 1cm Größer als Durchmesser der Pfannen-Form)
	\item Mit einer Gabel Löcher hineinstechen und wieder kalt stellen
	\item Äpfel schälen, vierteln und entkernen. Dann nochmals halbieren
	\item Zucker in einer ofenfesten Pfanne karamellisieren
	\item Butter zugeben und gut verrühren
	\item Pfanne von der Hitze nehmen und die Apfelspalten kreisförmig verteilen
	\item Ruhig sehr dicht legen (sie schrumpfen etwas ein)
	\item Pfanne wieder auf den Herd stellen und köcheln, bis der Saft der Äpfel eingedickt ist und sicht mit dem Karamell zu einer dicken Soße verbunden hat
	\item Teig auf die Äpfel legen
	\item In die Mitte 1-2 kleine Löcher stechen, damit der Dampf entweichen kann
	\item Im vorgeheizten Ofen bei 220\textdegree C Unter-Oberhitze für 15-20 Minuten backen
	\item Vor dem Stürzen etwa 30 Minuten abkühlen lassen
	\item Nochmals auf der Herdplatte so lange erhitzen, bis sich die Tarte bewegen lässt
	\item Mit einem Teller vorsichtig stürzen
	\item Tipp: Am besten schmeckt die Tarte Tatin warm mit Vanilleeis, Creme Fraiche oder auch einfach geschlagener Sahne.  
\end{enumerate}