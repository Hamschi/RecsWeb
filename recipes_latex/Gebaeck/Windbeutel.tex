\newpage
\titlelink{Windbeutel}{https://youtu.be/h0-izQRo8Xc?feature=shared}
\paragraph{Zutaten}
\begin{itemize}[noitemsep]
	\item 125ml Wasser
	\item 125ml Milch
	\item 125g Butter
	\item 5g Salz
	\item 10g Zucker
	\item 150g Weizenmehl
	\item 4 Eier
	\vspace{0.5cm}
	\item 200ml Sahne
	\item 1 Vanilleschote
	\item Puderzucker nach Geschmack
\end{itemize}


\paragraph{Zubereitung}
\begin{enumerate}[noitemsep]
	\item Ofen auf 230\textdegree C Ober-Unterhitze vorheizen
	\item Die Milch, das Wasser, die Butter und Salz und Zucker in einem Topf zum kochen bringen 
	\item Sobald die Butter vollständig aufgelöst ist: Das Mehl sieben und in die kochende Milch-Wasser-Mischung geben
	\item Mit einem Kochlöffel so lange rühren, bis sich der Teig gut von den Seiten des Topfes löst und sich ein weißer Belag bildet
	\item Den Teig in eine Schüssel geben und etwas abkühlen lassen
	\item  Eier nach und nach zum Teig geben, während mit einem Handrührgerät gut gerührt wird
	\item Teig in einen Spritzbeutel füllen und auf ein Blech mit Backpapier kleine Windbeutel aufspritzen
	\item Blech mit etwas Wasser auf den Boden des Ofens während des Backvorgangs legen
	\item Auf mittlere Schiene 5 Minuten backen
	\item Ofen auf 180\textdegree C runter drehen und für weitere 30 Minuten backen
	\item Danach: Ofen ausschalten und Windbeutel für 15 Minuten bei offenem Ofen auskühlen lassen
	\item Windbeutel anschließend auf einem Gitter auskühlen lassen
	\item Vanillemark auskratzen
	\item Sahne, Vanillemark und Puderzucker mit einem Rührgerät steif schlagen
\end{enumerate}
\placeimage{Bilder/Windbeutel}{11cm}{-5cm}{0.6}