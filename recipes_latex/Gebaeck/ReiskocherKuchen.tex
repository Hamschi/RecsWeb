\newpage
\titlelink{Bánh Bông Lan (not done yet)}{https://youtu.be/tdT0X1ImNas?feature=shared}

\paragraph{Zutaten}
\begin{itemize}[noitemsep]
	\item 6 Eier (ca. 65g pro Ei)
	\item 60ml Milch
	\item 120g Mehl
	\item 160g Zucker
	\item 60g Öl 
	\item 1/2 TL Vanillesirup
	\item 1 TL Limettensaft 
	\item 1 Prise Salz
	\item Note: Sie benutzt einen sehr großen Reiskocher
\end{itemize}


\paragraph{Zubereitung}
\begin{itemize}
\begin{enumerate}[noitemsep]
	\item Mehl sieben
	\item Eigelb vom Eiweiß trennen und in verschiedene Schüsseln geben
	\item Vanillesirup zum Eigelb geben
	\item Salz zum Eiweiß geben und mit einem Mixer schlagen bis Blasen erkennbar sind
	\item Limettensaft hinzugeben und das Eiweiß fertig schlagen 
	\item Sobald das Eiweiß so aussieht wie ein Schaumbad: Zucker in kleinen Abständen hinzugeben und fertig schlagen
	\item Eigelb hinzugeben und weiter mixen, bis alles einfarbig wird 
	\item Mehl hinzugeben, kurz unterrühren und anschließend mit dem Mixer gleichmäßig rühren
	\item Öl mit der Milch mischen 
	\item 2 Teigschaber Teig zur Öl-Milch-Mischung geben und umrühren
	\item Anschließend diese Mischung zum Teig unterrühren mit einem Teigschaber
	\item Seiten des Reiskochers mit Öl beschmieren und Mehl darüber träufeln
	\item Backpapier auf den Boden des Reiskochers legen
	\item Teig in den Reiskocher geben 
	\item Mit einem dünnen Stab sprudelförmig über den Teig gehen
	\item Reiskochtopf mehrfach auf eine Oberfläche schlagen, damit sich der Teig verteilt
	\item Damit der Kuchen kein Wasser abbekommt: Ein Tuch an den Reiskocher geben, bevor der Deckel zugemacht wird
	\item Reiskocher einschalten und 3 Minuten kochen lassen
	\item Danach für 15 Minuten auf der Wärmeseinstellung lassen
	\item Reiskocher wieder für 3 Minuten in den Kochmodus stellen
	\item Okay, den letzten Teil habe ich nicht so ganz verstanden, aber iwie geht der Prozess so ne Stunde
	\item Kuchen anschließend abkühlen lassen und anschließend aus dem Topf nehmen
\end{enumerate}