\clearpage
\titlelink{Pappa Roti/ Roti Boy (6 Stück) (not done yet)}{https://youtu.be/xL8LeyDz2mk?si=7pY9SRCgXHF159cP}

\paragraph{Zutaten}
\begin{itemize}[noitemsep]
	\item 200g Brotmehl
	\item 40g Zucker
	\item 4g Salz
	\item 35g Milchpulver
	\item 90g Wasser (lauwarm)
	\item 1 Ei (groß, ca. 50g ohne Schale)
	\item 7g Trockenhefe 
	\item 20g Butter (Raumtemperatur)
	\vspace{0.5cm}
	\item 67g Butter (Raumtemperatur)
	\item 80g Zucker
	\item 1 Ei (groß)
	\item 3 TL Instantkaffee 
	\item 7g Wasser (kochend)
	\item 85g Mehl
	\item 60g Butter (gesalzen, kühl)
\end{itemize}


\paragraph{Zubereitung}
\begin{enumerate}[noitemsep]
	\item Brotmehl,  Zucker, Salz, Milchpulver, Wasser, Eier und die Trockenhefe mit Hilfe eines Knethakens 8-10 Minuten verrühren bis der Teig elastisch ist und etwas \glqq{}Strength\grqq{} hat
	\item Butter hinzufügen und für weitere 5 Minuten kneten bis der Teig den Windowplane-Test erfüllt
	\item Teig zu einer Kugel formen und in einer Schüssel für ca. 1h ruhen lassen, bis sich der Teig ungefähr verdoppelt hat
	\item Die 60g Butter in 6 Stücke à 10g unterteilen und in den Kühlschrank stellen
	\item Sobald der Teig fertig gegangen ist: In 6 gleichschwere Stücke unterteilen
	\item Jedes dieser Teige in zwei Teile unterteilen
	\begin{description}[noitemsep, nolistsep]
		\item \rarrow Ein Teil entspricht 1/3 des Teiges, der andere Teil 2/3 des Teiges
	\end{description}
	\item Den kleineren Teil des Teiges nochmals kneten und damit ein Stück Butter ummanteln
	\item Mit dem größeren Teil des Teiges nun  den kleineren Teil des Teiges ummanteln (sollte kugelförmig enden)
	\item Die letzten Schritte für alle Teigstücke wiederholen
	\item Backpapier auf ein Backblech auslegen und die Teigkugeln darauf verteilen (je 10-12cm Abstand)
	\item Die Teige abdecken und für 1-1.5 Stunden ruhen lassen bis sie ungefähr doppelt so groß sind
	\item Instantkaffee im kochenden Wasser auflösen lassen
	\item Butter und Zucker in einer Schüssel zu einer cremigen Masse verrühren
	\item Eier und Kaffee hinzufügen und zu einer fluffigen Konsistenz verrühren
	\item Mehl unterrühren bis alles zu einer Masse wird
	\item Diese Masse in einen Spritzbeutel geben und zur Seite legen, bis sie benötigt wird
	\item Sobald die Teige aufgegangen sind: Mit dem Spritbeutel das Topping gleichmäßig auf die Brötchen verteilen
	\begin{description}[noitemsep, nolistsep]
		\item \rarrow Mind. 3/4 der Oberfläche eines Buns sollte abgedeckt sein
	\end{description}
	\item Die Rotiboys in den auf 180\textdegree C vorgeheizten Ofen stellen (Over-/Unterhitze)
	\item Hitze sofort auf 140\textdegree C reduzieren und für 10 Minuten stehen lassen
	\item Danach: Hitze auf Grillfunktion stellen und die Temperatur maximieren
	\item Nach ca. 3 Minuten entnehmen und abkühlen lassen
\end{enumerate}