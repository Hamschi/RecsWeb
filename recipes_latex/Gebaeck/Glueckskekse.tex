\clearpage
\titlelink{Glückskekse (10-12 Stück) (not done yet)}{https://youtu.be/P6CgCL2HqI0?si=eABYdW0aOGFdQEu9}

\paragraph{Zutaten}
\begin{itemize}[noitemsep]
	\item 1 Eiweiß (groß)
	\item 50g Zucker
	\item 1 EL Milch
	\item 1/8 TL Vanilleextrakt
	\item 1/8 TL Mandelextrakt
	\item 1 Prise Salz
	\item 1 EL Butter 
	\item 42g Mehl
\end{itemize}

\paragraph{Zubereitung}
\begin{enumerate}[noitemsep]
	\item Butter schmelzen
	\item Alle Zutaten mit Ausnahme des Mehls in eine Schüssel geben und zu einer Masse verrühren
	\item Mehl anschließend dazugeben und zu einem Teig verrühren
	\item Backpapier auf einem Backblech auslegen
	\item 1 EL nehmen und auf das Blech platzieren
	\item Mit dem Löffel mit Hilfe von kreiselnden Bewegungen den Teig ausbreiten (ca. 10cm Durchmesser)
	\item Bei 180\textdegree C Ober-/Unterhitze für 10 Minuten in den vorgeheizten Ofen geben
	\begin{description}[noitemsep, nolistsep]
		\item \rarrow ca. 50-75\% des Kekses sollte angebräunt sein
	\end{description}
	\item Mit Hilfe eines Pfannenwenders die Kekse vom Boden lockern
	\item Notizzettel mittig auf die Kekse platzieren
	\item Keksteige mittig zuklappen
	\begin{description}[noitemsep, nolistsep]
		\item \rarrow Sollten die Kekse zu fest geworden sein: Wieder für wenige Minuten in den Ofen tun 
	\end{description}
	\item Kekse an der geschlossenen Kante herunterdrücken und in eine kleine Ofenform geben, um sie zu stabilisieren
	\begin{description}[noitemsep, nolistsep]
		\item \rarrow Man kann diese Kante auch gegen die Kante vom Blech drücken
	\end{description}
	\item Abkühlen lassen
\end{enumerate}