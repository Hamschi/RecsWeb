\clearpage
\titlelink{Apfelstrudel (5-6 Portionen) (not done yet)}{https://youtu.be/izf7yU-FK0c?si=EMmJ996jMx4k_RPy}

\paragraph{Zutaten}
\begin{itemize}[noitemsep]
	\item 150g Mehl Typ 550
	\item 75g Wasser (lauwarm)
	\item 15g Pflanzenöl
	\item 1/4 TL Salz
	\item 1 Eigelb (Größe M)
	\vspace{0.5cm}
	\item 1.2kg Äpfel (ca. 750g geschält ohne Kerngehäuse, z.B.  Braeburn, Cox Orange, Jonagold)
	\item 50g Rosinen
	\item 50g Semmelbrösel
	\item 60g Zucker
	\item 1 TL Zimt
	\item 60g Butter
	\item Puderzucker zum Bestäuben
\end{itemize}

\paragraph{Zubereitung}
\begin{enumerate}[noitemsep]
	\item Mehl, Wasser, Pflanzenöl, Salz und Eigelb in eine Schüssel geben und mit einem Holzlöffel verrühren bis das Mehl gebunden ist
	\item Teig auf eine Arbeitsfläche geben und mit der Hand gut durchkneten, bis ein weicher glatter Teig entsteht (ca. 5-10 Minuten)
	\item Teig in eine Folie einschlagen und für 1 Stunde bei Zimmertemperatur gehen lassen
	\item Währenddessen: Äpfel waschen, schälen, vierteln und das Kerngehäuse entfernen
	\item Apfelstücke in etwa 2-3mm dicke Spalten schneiden und in eine Schüssel geben
	\item Äpfelspalten mit Rosinen vermischen und beiseite stellen
	\item Sobald der Teig fertig gegangen ist: auf eine Arbeitsfläche geben und mit Mehl bestäuben
	\item Mit einem Nudelholz ausrollen
	\item Ein ausreichend großes Tuch auslegen, großzügig mit Mehl bestäuben und den Strudelteig darauf legen
	\item Vorsichtig den Teig auf allen Seiten in die Länge ziehen
	\begin{description}[noitemsep, nolistsep]
		\item \rarrow man sollte hindurchsehen können
		\item ca. 45cm x 60cm
	\end{description}
	\item Butter auf dem Herd zum Schmelzen bringen und den Teig damit einpinseln
	\item Brösel darauf verteilen
	\item Sollten Brösel übrig bleiben: Brösel in die Apfelmischung hineingeben (optional)
	\item Äpfel großzügig verteilen
	\begin{description}[noitemsep, nolistsep]
		\item \rarrow Am Rand und oben jeweils etwas Platz lassen
	\end{description}
	\item Den Zucker mit dem Zimt mischen und auf den Äpfeln verteilen
	\item ggf. etwas dickere Ränder abschneiden
	\item Den Strudel unten ein paar Zentimeter einschlagen, dann mit Hilfe des Tuches aufrollen
	\item Backblech mit etwas Butter einpinseln (alternativ Backpapier auslegen)
	\item Den Strudel auf das Blech geben, mit Butter einpinseln und im vorgeheizeten Backofen bei 200 \textdegree C Ober-/Unterhitze für 30-35 Minuten backen
	\item  Den gebackenen Apfelstrudel herausnehmen, nochmals mit Butter einpinseln und etwa 30 Minuten abkühlen lassen
	\item Mit Puderzucker bestäuben und am besten noch lauwarm servieren
\end{enumerate}