\newpage
\titlelink{Matcha Cheesecake (need pic)}{https://youtu.be/X0nbs5x0hMI?si=mJSTia-4VQzpiWAY}

\paragraph{Zutaten}
\begin{itemize}[noitemsep]
	\item 100g Butterkekse
	\item 40g Butter (geschmolzen)
	\item 5-6g Gelatinepulver
	\item 25ml Wasser
	\item 200g Frischkäse (Raumtemperatur)
	\item 100g Joghurt
	\item 65g Zucker
	\item 200g Schlagsahne
	\item 7g Matcha
	\item 20ml Milch (heiß)
	\item Lebensmittelfarbe (grün, optional)
\end{itemize}


\paragraph{Zubereitung}
\begin{enumerate}[noitemsep]
	\item Kekse in einem Gefrierbeutel in Krümmel zerschlagen
	\item Butter (geschmolzen) dazugeben und alles gründlich durchkneten
	\item Eine 15cm-Durchmesser Tortenform auf einem Tortentablett legen und mit den Krümmeln einen Boden machen 
	\begin{description}[noitemsep, nolistsep]
		\item \rarrow Die Oberfläche sollte flach sein, aber nicht zu fest andrücken, sonst wird es nach dem Abkühlen zu hart sein
	\end{description}
	\item Den Keksboden 20 Minuten in den Kühlschrank stellen
	\item Wasser zur Gelatine in einem kleinen Behälter geben und 15 Minuten ziehen lassen 
	\item Den Frischkäse zu einer glatten Creme rühren
	\item Joghurt und Zucker hinzugeben und zu einer geschmeidigen Masse umrühren
	\item Den Behälter mit der Gelatine in heißes Wasser geben
	\item Umrühren, damit die Gelatine schmilzt und sich auflöst
	\item Gelatine zum Frischkäse geben und darin auflösen lassen
	\item Schlagsahne in eine Schüssel kippen und schlagen bis es eine 70-80\% Konsistenz erreicht hat
	\item Diese Sahne nun zum Frischkäse geben und leicht unterrühren
	\begin{description}[noitemsep, nolistsep]	
		\item \rarrow Nicht zu stark umrühren!
	\end{description}
	\item Die Hälfte davon nun gleichmäßig über den Keksboden verteilen
	\item Alles wieder für weitere 20 Minuten in den Kühlschrank tun
	\item Matcha in der heißen Milch auflösen lassen, sodass keine Klumpen entstehen
	\item Sobald die Flüssigkeit abgekühlt ist, die Matchamilch zum Frischkäse geben
	\begin{description}[noitemsep, nolistsep]
		\item \rarrow Auch hier gilt: schnell und vorsichtig falten und nicht großflächig mischen
	\end{description}
	\item Ggf. grüne Lebensmittelfarbe dazugeben, wenn die Farbe nicht zusagt
	\item Den grünen Teil nun ebenfalls in die Form geben und gleichmäßig verteilen
	\begin{description}[noitemsep, nolistsep]
		\item \rarrow Für ein schönes Ergebnis muss die Oberfläche glatt sein
	\end{description}
	\item Nach 3-4h im Kühlschrank ist die Torte auch fertig, sobald die Form entnommen wurde
\end{enumerate}
