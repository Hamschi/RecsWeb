\newpage
\titlelink{Mango Cheesecake (not done yet)}{https://youtu.be/E9gvM522EqI?si=ky9FkWxZ2P4jrnbv}

\paragraph{Zutaten}
\begin{itemize}[noitemsep]
	\item 120g Butterkekse
	\item 60g Butter (geschmolzen)
	\item 3 Mangos (ca. 700-800g)
	\item 7g Gelatine
	\item 35ml Wasser
	\item 250g Frischkäse
	\item 60g Zucker
	\item 10ml Zitronensaft
	\item 200g Mangopüree
	\item Wasser (heiß)
	\item 150g Schlagsahne
	\item 8g Gelatine
	\item 40ml Wasser
	\item 50ml Wasser 
	\item 200g Mangopüree
	\item 25g Zucker
	\item Minzblätter zum Verzieren
\end{itemize}

\paragraph{Zubereitung}
\begin{enumerate}[noitemsep]
	\item Kekse in einem verschließbaren Beutel möglichst klein zerbröseln 
	\item Butter hinzugeben und alles möglichst gleichmäßig vermengen
	\item Keksmasse in eine 15cm-Form geben, zu einen Keksboden formen und in den Kühlschrank geben
	\item Mangosfleisch entnehmen und mit einem Pürierstab pürieren
	\item Wasser zur Gelatine geben und 15 Minuten warten
	\item Frischkäse und Zucker in einer Schüssel vermengen bis der Frischkäse weich und der Zucker aufgelöst ist
	\item Zitronensaft und 200g Mangopüree untermischen
	\item Die Gelatineschüssel in heißes Wasser stellen, sodass sie schmilzt and anschließend ebenfalls untermischen
	\item Schlagsahne zu 80\% schlagen und untermischen
	\item Die Masse komplett in die Tortenform geben und 20 Minuten kühl stellen
	\item Gelatine mit Wasser kombinieren und 15 Minuten abwarten
	\item Wasser, 200g Mangopüree und Zucker in einen kleinen Topf geben 
	\item Topf auf den Herd geben und zum Kochen bringen
	\begin{description}[noitemsep, nolistsep]
		\item \rarrow Währenddessen kontinuierlich rühren 
	\end{description}
	\item Sobald es kocht: In einen Messbecher geben 
	\item Gelatine in den Messbecher geben und auflösen lassen 
	\item Nachdem die Masse abgekühlt ist: eine 1cm-Schicht davon über die Torte geben
	\item Torte für 4-6h in den Kühlschrank legen
	\item Die restliche Masse in Halbkugel-Eisformen geben 
	\item Torte aus der Form entnehmen
	\item Die Gelatine aus den Eisformen und den Pfefferminz zum Dekorieren des Kuchens nutzen
\end{enumerate}