\newpage
\subsection{Chocolate Chip Cookies}
\paragraph{Zutaten}
\begin{itemize}[noitemsep]
	\item 112g Butter 
	\item 1/2 TL koscheres Salz 
	\item 1/2 TL Baking Soda
	\item 200g Schokolade (zartbitter)
	\item 200g Zucker (braun)
	\item 112g Zucker (weiß)
	\item 1 Ei 
	\item 1 Eigelb 
	\item 2 TL Vanilleexterakt 
	\item 200g Mehl
	\item Meersalz
\end{itemize}
\paragraph{Zubereitung}
\begin{enumerate}[noitemsep]
	\item Mehl, Salz, und Baking Soda in einer Schüssel verrühren 
	\item Schokolade grob hacken 
	\item Butter in der Pfanne schmelzen lassen
	\item Zucker (braun) und Zucker (weiß) in einer Schüssel mischen
	\item Anschließend die geschmolzene Butter dazukippen 
	\item ein Ei dazugeben und bei mittel-schwacher Mixerstärke mixen 
	\item nun das Eigelb sowie das Vanilleextrakt dazugeben 
	\item Mehl unterrühren 
	\item gehackte Schokolade mit einem Schaber unterrühren
	\item den Cookie-Teig mit einer Frischhaltefolie überdecken und 30-45 Minuten ruhen lassen
	\begin{description}[noitemsep,nolistsep]
		\item $\rightarrow$ Frischhaltefolie wird an den Cookie-Teig gedrückt
		\item $\rightarrow$ Je länger die Ruhezeit, desto besser (2-3 Tage wären noch nicer)
	\end{description}
	\item Backofen auf 180\textdegree C Ober-/Unterhitze vorheizen
	\item Runde Kugeln formen (z.B. mit einem Cookieportionierer) und im großen Abstand auf das Backblech verteilen
	\item Cookies für 8-10 Minuten im Ofen backen 
	\item optional: nach 6 Minuten 2-3x Pan Bake Method und nach weiteren 2-3 Minzuten danach
	\item Nachdem die aus dem Ofen kommen: mit etwas Meersalz bestäuben
\end{enumerate}
\placeimage{Bilder/Cookies}{11cm}{-5.2cm}{0.65}