\clearpage
\titlelink{bánh bò nướng (not done yet)}{https://youtu.be/jKvPdcpGatc?si=qhG6Pljrt0qOOrMi}
\label{BanhBoNuong}

\paragraph{Zutaten}
\begin{itemize}[noitemsep]
	\itemm 160g Zucker
	\item 250ml Kokosmilch
	\item 1 Prise Salz
	\item 190 Tapiokamehl
	\item 10g Mehl
	\item 5 Eier (groß)
	\item 10g Backpulver
	\item 1 Pck. Vanillezucker 
	\item 1 Schuss Pandan-Extrakt
	\item 10g Butter (geschmolzen) (alternativ: Öl)
\end{itemize}

\paragraph{Zubereitung}
\begin{enumerate}[noitemsep]
	\item Ofen auf 200\textdegree C vorheizen 
	\item Sobald vorgeheizt: Eine 22cm-Durchmesser Kuchenform für mindestens 5 Minuten reinlegen
	\item Zucker und Kokosmilch in einer einem Topf über dem Herd bei mittlerer Hitze vermengen
	\item Sobald der Zucker sich aufgelöst hat: Herd ausschalten und Salz unterrühren
	\item Eier in einer Schüssel schlagen, sodass es eine Masse wird
	\item Kokosmasse zum Ei unterrühren
	\item Tapiokamehl unterrühren
	\item Mehl, Vanillezucker und Panda-Extrakt unterrühren
	\item Teig sieben
	\item Backpulver unterrühren
	\item Öl unterrühren
	\item Form mit Butter beschmieren
	\item Den Teig durch einen Plastik-Seier mit vielen Löchern in die Form gießen
	\item Kuchenform mit Alufolie abdecken und in den Ofen für 35 Minuten bei 170\textdegree C in den Ofen geben
	\item Alufolie entfernen und für weitere 15 Minuten bei 165\textdegree C backen
	\item Ofen abschalten und für 10 Minuten im Ofen ruhen lassen
	\item Anschließend aus dem Ofen nehmen und Kopfüber (z.B. über einen Teller) aufstellen zum Abkühlen
	\item Kuchen aus der Form nehmen und zurecht schneiden
\end{enumerate}