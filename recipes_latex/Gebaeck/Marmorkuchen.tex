\newpage
\subsection{Marmorkuchen (not done yet)}
\paragraph{Zutaten}
\begin{itemize}[noitemsep]
	\item 200g Butter (weich)
	\item 160g Zucker (noch nicht reduziert)
	\item 1 Pck. Vanillezucker
	\item 1 Msp. Zitronenabrieb
	\item 2 EL Rum
	\item 6 Eier
	\item 1 Prise Salz
	\item 120g Zucker (noch nicht reduziert)
	\item 280g Mehl
	\item 1/2 Pck. Backpulver
	\item 100ml Milch (lauwarm)
	\item 20g Kakaopulver 
	\item Puderzucker
	\item Butter und Mehl für die Form
\end{itemize}
\paragraph{Zubereitung}
\begin{enumerate}[noitemsep]
	\item die Butter mit Zucker und Vanillezucker cremig rühren und dann den Zitronenabrieb und den Rum unterrühren
	\item die Eier trennen und die Eigelbe einzeln nacheinander in die Butter-Zucker-Masse rühren 
	\item die 6 Eiweiße mit dem Salz halbfest schlagen und mit dem restlichen Zucker zu Schnee schlagen
	\item Mehl mit Backpulver mischen und leicht anwärmen
	\item abwechselnd (etwa in 3-4 Schritten) zuerst etwas Mehl auf die Eimasse sieben, dann etwas lauwarme Milch dazugießen und eine Portion Eischnee darauf geben
	\item mit einem Holzlöffel (idealerweise mit Loch in der Mitte) alles unterheben
	\item die Kuchenform mit Butter ausstreichen und mit Mehl bestäuben
	\item die Hälfte des Teiges in die Form füllen
	\item bei der anderen Hälfte den Kakaopulver in den Teig sieben
	\item den dunklen Teig über den hellen Teig verteilen und mit einer Gabel spiralförmig unterziehen 
	\begin{description}[noitemsep, nolistsep]
		\item $\rightarrow$ mit einer großen Fleisch-/ Grillgabel, die man leicht schräg hält, geht es besonders leicht
	\end{description}
	\item im vorgeheizten Backofen ca. 60 Minuten backen (180 \textdegree C Ober-/Unterhitze)
	\begin{description}[noitemsep, nolistsep]
		\item $\rightarrow$ er soll keine zu harte Kruste kriegen und eher heller backen
		\item Stäbchenprobe!
	\end{description}
	\item Danach: den Kuchen in der Form leicht auskühlen lassen und dann auf ein Kuchengitter stürzen
	\item den Kuchen dann mit Puderzucker beziehen
\end{enumerate}