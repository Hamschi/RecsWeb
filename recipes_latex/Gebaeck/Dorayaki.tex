\newpage
\subsection{Dorayaki}
\paragraph{Zutaten}
\begin{itemize}[noitemsep]
	\item 150g Mehl
	\item 1 TL Baking Soda
	\item 2 Eier
	\item 112,5g Zucker 
	\item 1 EL Honig
	\item 180ml Milch
	\item Anko
\end{itemize}
\paragraph{Zubereitung}
\begin{enumerate}[noitemsep]
	\item Mehl und Baking Soda in einer Schüssel vermischen
	\item In einer anderen Schüssel: Eier, Zucker und Honig miteinander vermischen
	\item Milch hinzufügen und gründlich vermischen
	\item die trockenen Zutaten in die Ei-Mischung unterrühren
	\item Schneebesen benutzen bis der Teig sehr glatt ist
	\item ein non-stick frying pan bei mittlerer/hoher Flamme erwärmen
	\item Währenddessen: Öl in der Pfanne mit einer Serviette verteilen
	\item Überschüssiges Öl mit einer Serviette abwischen
	\item medium-low heat
	\item 1/8 cup vom Teig in die Pfanne
	\begin{description}[noitemsep, nolistsep]
		\item $\rightarrow$ Form wie die eines Pancakes
	\end{description}
	\item drauf lassen bis sich ein paar Löcher bilden und die Kanten trocken werden (ca. 2 Minuten)
	\item flippen und eine weitere Minute auf dem Herd lassen
	\item fertige Pancakes auf einen Teller tun und mit einer nassen Serviette bedecken
	\begin{description}
		\item $\rightarrow$ bis alle Pancakes fertig sind
	\end{description}
	\item ein Pancake in die Hand nehmen und einen gehäuften EL Anko mittig legen
	\item anderes Pancake als Deckel benutzen
	\item gesamten Dorayaki in Frischhaltefolie einpacken und leicht drücken 
	\item Kanten auch zusammendrücken
\end{enumerate}
\placeimage{Bilder/Dorayaki}{11.5cm}{-4cm}{0.5}