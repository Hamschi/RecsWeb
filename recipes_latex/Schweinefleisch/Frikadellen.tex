\newpage
\titlelink{Frikadellen (not done yet)}{https://youtu.be/5n272Moqlyg?si=0Cr2B3LrmVXW8CzW}
\paragraph{Zutaten}
\begin{itemize}[noitemsep]
	\item 800g Hackfleisch (gemischt)
	\item 1 Zwiebel
	\item 1 Brötchen vom Vortag
	\item Milch
	\item 1-2 Eier
	\item 2 TL Paprikapulver
	\item 2 TL Ketchup
	\item 1 TL Senf
	\item 1 EL Petersilie
	\item 1 TL getrockneter Thymian (alternativ Majoran)
	\item 1.5 TL Salz
	\item Pfeffer
\end{itemize}
\paragraph{Zubereitung}
\begin{enumerate}[noitemsep]
	\item Brötchen halbieren und in der Milch einweichen (muss ggf. später darin gewendet werden)
	\item Zwiebeln in kleine Würfeln schneiden und in Öl anschwitzen
	\item Sobald die Zwiebeln weich sind: Thymian dazugeben und Zwiebeln abkühlen lassen
	\item Blattpetersilie hacken
	\item Sobald das Brötchen weich geworden ist: Milch ausdrücken
	\item Salz, Paprikapulver, Ketchup, Senf, Ei(er), Zwiebeln, Petersilie, Brötchen dazugeben und alles durchkneten
	\item Frikadellen für eine halbe Stunde in den Kühlschrank stellen
	\item Danach: Fleischmasse aus dem Kühlschrank nehmen und zu Frikadellen formen
	\item Frikadelle in einer heißen Pfanne anbraten (Öl ist nicht unbedingt nötig)
	\item Sobald die Frikadellen Farbe haben: Frikadellen umdrehen und Temperatur auf mittlere Hitze reduzieren
	\begin{description}[noitemsep, nolistsep]
		\item \rarrow Sollten die Frikadellen von beiden Seiten schon Farbe haben: Deckel auf die Pfanne
	\end{description}
	\item Frikadellen sind gar, wenn beim Andrücken mit dem Finger die Frikadellen nicht so weich sind
\end{enumerate}