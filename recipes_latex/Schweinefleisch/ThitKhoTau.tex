\newpage
\selectlanguage{vietnamese}
\subsection{Thịt kho Tầu (2 Portionen)}
\selectlanguage{ngerman}
\paragraph{Zutaten}
\begin{itemize}[noitemsep]
	\item 2 TL Salz
	\item 500ml heißes Wasser
	\item 6 Eier
	\item 1 Schuss Essig
	\item 400g Schweinebauch
	\item 1 TL Salz
	\item 1 TL Zucker
	\item 1 TL Hühnerbrühe
	\item 1 EL Fischsoße
	\item 1 TL Sojasoße (dunkel)
	\item 2 Zehen Knoblauch
	\item 1/2 Zwiebel
	\item 300-400ml kochendes Wasser
	\item 2 TL Zucker
	\item 1 Chilischote (klein)
	\item 1 Zehe Knoblauch
\end{itemize}
\paragraph{Zubereitung}
\begin{enumerate}[noitemsep]
	\item Eier hartkochen und anschließend pellen
	\item Salz, Wasser, Essig miteinander mischen und das Fleisch darin waschen
	\item Salz, Hühnerbrühe, Zucker, Fischsoße und die dunkle Sojasoße in einer Schüssel zu einer Soße mischen 
	\item Fleisch in mundgerechte Stücke oder Riesenbrocken schneiden
	\item Knoblauch kleinschneiden und das Fleisch darin braten
	\item Marinade gleichmäßig auf das Fleisch verteilen und mindestens 1h ziehen lassen
	\item Danach: Knoblauch und Zwiebeln zerkleinern, Wasser kochen
	\item Öl und Zucker in einen Topf geben und den Zucker im Topf bei hoher Flamme erhitzen, bis es zu brodeln beginnt
	\begin{description}[noitemsep, nolistsep]
		\item $\rightarrow$ dabei kontinuierlich umrühren
	\end{description}
	\item Sobald ein süßer Geruch wahrzunehmen ist: Flamme auf 7-8 senken
	\item Sobald der Zucker brodelt: Feuer komplett ausmachen
	\item Zwiebeln und Knoblauch in den Zucker geben und darin wenden
	\item Fleisch dazugeben und gut verrühren
	\item den Herd wieder hoch auf 8 stellen
	\item dieses Wasser in die Pfanne geben und das Feuer auf mittlere Hitze senken (6)
	\item Eier und Chilischote in den Topf tun und das Fleisch für 1,5-2h köcheln lassen
\end{enumerate}
\placeimage{Bilder/ThitKhoTau}{11.5cm}{-4.5cm}{0.65}