\newpage
\subsection{Schweinefleisch süß sauer nach Mamas Art}
\paragraph{Zutaten}
\begin{itemize}[noitemsep]
	\item 500g Schweinebauch
	\item 1/4 Zwiebel
	\item 1 Zehe Knoblauch
	\item 1 Chilischote
	\item 1 EL Zucker
	\item 1 EL Essig
	\item 1 TL Salz
	\item 2 EL Fischsoße
	\item 2 EL Speisestärke
	\vspace{0.5cm}
	\item 250ml Wasser
	\item 1 EL Zucker
	\item 1 EL Essig
	\item 1 EL Fischsoße
	\item 1/2 EL dunkle Sojasauce
	\item 1/2 Speisestärke
\end{itemize}
\paragraph{Zubereitung}
\begin{enumerate}[noitemsep]
	\item Fleisch in mundgerechte Stücke schneiden und saubermachen
	\item Zwiebeln, Knoblauch und Chili kleinschneiden und zum Fleisch geben
	\begin{description}[noitemsep, nolistsep]
		\item $\rightarrow$ etwas davon für später zur Seite legen
	\end{description}
	\item Fischsoße, Salz, Zucker und Essig dazugeben, dann mischen
	\item Speisestärke dazugeben und vermischen
	\item Fleisch ca. 1h einziehen lassen
	\item Fleisch in einer heißen Pfanne mit Öl anbraten 
	\begin{description}[noitemsep, nolistsep]
		\item $\rightarrow$ später Herdestufe auf eine mittlere Hitze senken
	\end{description}
	\item Sobald das Fleisch goldbraun ist: wenden
	\item Sobald beide Seiten goldbraun sind: Fleisch aus der Pfanne holen
	\item Währenddessen: Wasser in den vorherigen Behälter mit dem Fleisch tun
	\item dann die anderen Gewürze hinzufügen 
	\item Im restlichen Öl die zur Seite gelegten Chili, Knoblauch und Zwiebeln anbraten, bis es aromatisch wird
	\item die Flüssigkeit reinschütten
	\item Sobald das Wasser brodelt: Fleisch hinzugeben
	\item Fertig, sobald die Flüssigkeit zum großen Teil verdampft und sehr dickflüssig wird
\end{enumerate}
\placeimage{Bilder/Schwein_suesssauer}{10cm}{-5cm}{0.6}