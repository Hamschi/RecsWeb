\newpage
\subsection{Tonkatsu (2 Portionen)}
\paragraph{Zutaten}
\begin{itemize}[noitemsep]
	\item 1 Eisbergsalat
	\item 1 Zitrone
	\item 1 Tomate
	\item 3 Stück Schweinerücken (ca. 500g?) (alternativ Schulter)
	\item Salz und Pfeffer
	\item 1 Ei
	\item 4 EL Mehl
	\item 4 EL Wasser
	\item 1 EL Öl
	\item Panko
	\item Tonkatsu Soße
\end{itemize}


\paragraph{Zubereitung}
\begin{enumerate}[noitemsep]
	\item Eisbergsalat in dünne Streifen schneiden
	\item Zitrone und Tomate in Spalten schneiden 
	\item das Fleisch plattdrücken (ca. 1.3cm dick)
	\item anschließend das Fleisch in ihre ursprüngliche Größe formen 
	\vspace{0.5cm}
	\item Fleisch auf beiden Seiten mit Salz und Pfeffer würzen 
	\item Fleisch über Nacht in den Kühlschrank legen 
	\item eine Panierstraße bilden: Mehl, geschlagenes Ei, Panko 	
	\begin{description}[noitemsep, nolistsep]
		\item $\rightarrow$ wenn es vollständig mit Panko bedeckt wird, das Fleisch andrücken
	\end{description}
	\item Fleisch für 5 Minuten ruhen lassen 
	\item Das Fleisch auf beiden Seiten je 2 Minuten bei 170\textdegree C frittieren 
	\item Teller mit sehr viel Salat, und je einer Zitronen- und Tomatenspalte belegen 
	\item Das Fleisch auf beiden Seiten je für wenige Minuten erneut frittieren, sodass die Kruste eine goldbraune Farbe annimmt (190\textdegree C)
	\item Sobald das Fleisch aus dem Öl genommen wird: leicht mit Salz bestreuen
	\item Das Fleisch in Streifen schneiden 
	\item Das Fleisch auf dem vorbereiteten Teller legen und mit Tonkatsusoße servieren
\end{enumerate}
\placegraphic{Bilder/Tonkatsu}{0.5}