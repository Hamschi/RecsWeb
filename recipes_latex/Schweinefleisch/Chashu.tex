\newpage
\subsection{Chashu}
\paragraph{Zutaten}
\begin{itemize}[noitemsep]
	\item 900-1100g Schweinebauch am Stück 
	\item Faden
	\item Öl zum Braten
	\item 3 Frühlingszwiebeln
	\item 15-20g Ingwer
	\item 240ml Sake
	\item 240ml Sojasoße 
	\item 150g Zucker 
	\item 480ml Wasser 
\end{itemize}
\paragraph{Zubereitung}
\begin{enumerate}[noitemsep]
	\item Schweinefleisch zusammenrollen und mit Faden festigen 	
	\item Fleisch bei hoher Flamme im Öl braten bis alle Seiten goldbraun sind
	\begin{description}[noitemsep, nolistsep]
		\item $\rightarrow$ angefangen bei der Seite mit Haut
		\item $\rightarrow$ Fleisch drehen, sobald die untere Seite goldbraun ist
	\end{description}
	\item Währenddessen den grünen Teil der Zwiebeln entnehmen und den Ingwer in dünne Scheiben schneiden
	\item in einem Topf Sake, Sojasoße, Zucker, Wasser, Ingwer und die Frühlingszwiebeln  zum Kochen bringen  
	\item Sobald es brodelt: Fleisch in den Topf geben
	\begin{description}[noitemsep, nolistsep]
		\item $\rightarrow$ ggf. Dreck entfernen
	\end{description}
	\item Herd auf geringe/mittlere Flamme einstellen
	\item Alufolie dick in Form des Topfes schneiden (zweischichtig) und 6 Löcher (d=1cm) reinstechen (otoshibuta Ersatz)
	\item alle 30 Minuten das Fleisch wenden und das 2 Stunden lang 
	\item Danach: Herd ausschalten und Fleisch 10 Minuten kühlen lassen 
	\item Fleisch in eine Tüte legen 
	\item die Soße sieben und zwei Schöpfkellen Soße mit in die Tüte geben 
	\item Tüte so verschließen, dass möglichst keine Luft mehr in der Tüte bleibt 
	\item Chashu und die Soße über die Nacht in das Tiefkühlfach stellen  
	\item Fleisch aus der Tüte holen, Fäden entfernen 
	\item Fleisch nochmal anbraten (die haben nen Flammenwerfer benutzt)
	\begin{description}[noitemsep, nolistsep]
		\item $\rightarrow$ anderes Tutorial meint: nur eine  Seite knusprig anbraten
	\end{description}
\end{enumerate}
\placeimage{Bilder/Chashu}{11.5cm}{-4cm}{0.65}