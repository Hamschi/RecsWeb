\clearpage
\selectlanguage{vietnamese}
\titlelink{Hạt Lựu (not done yet)}{https://youtu.be/7C6Bj4f2Yv0?si=Vjk02NsZy2ncugYQ}
\selectlanguage{ngerman}
\label{HatLuu}

\paragraph{Zutaten}
\begin{itemize}[noitemsep]
	\item 150g Wasser-Kastanien
	\item 1 TL Lebensmittelfarbe (rot)
	\item 1 TL Wasser
	\item 100g Tapiokamehl
	\item 1 EL Litschisirup
	\item 1 EL Toddy Palmensamensirup
\end{itemize}


\paragraph{Zubereitung}
\begin{enumerate}[noitemsep]
	\item Wasser-Kastanien in kleine Würfel hacken
	\item Wasser-Kastanien mit Lebensmittelfarbe einfärben
	\item Wasser zu den gehackten Kastanien geben und gleichmäßig verteilen
	\item 15 Minuten ruhen lassen
	\item Anschließend kleine Mengen Tapiokomehl hinzugeben und gleichmäßig verteilen bis das Mehl aufgebraucht ist
	\begin{description}[noitemsep, nolistsep]
		\item \rarrow Sollten so ne Schicht Mehl außen haben, relativ trocken
	\end{description}
	\item Die Kastanienstücke nun in eine Sieb geben und überschüssiges Mehl abklopfen
	\item Kaltes Wasser in einer Schüssel vorbereiten und einen Topf mit Wasser erhitzen
	\item Sobald das Wasser kocht: die Kastanien dazugeben und bei mittlerer Hitze kochen
	\item Sobald die Kastanien anfangen an der Oberfläche zu schwimmen: eine weitere Minute kochen lassen und anschließend aus dem Wasserfischen
	\item Kastanien im kalten Wasser abschrecken
	\begin{description}[noitemsep, nolistsep]
		\item Sollte das Wasser lauwarm werden: Wasser auswechseln 
	\end{description}
	\item Sobald abgekühlt: in eine Tasse geben
	\item Litschi- und Palmsirup hinzugeben und umrühren
\end{enumerate}