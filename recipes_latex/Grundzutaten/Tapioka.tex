\newpage
\titlelink{Tapiokaperlen (not done yet)}{https://youtu.be/oAPAt-jSYcw?si=Db8mPi2QoQsHiQtg}
\paragraph{Zutaten}
\begin{itemize}[noitemsep]
	\item 200g Tapiokamehl
	\item 5g Kakao
	\item 20g Zucker 
	\item 50ml Wasser
\end{itemize}


\paragraph{Zubereitung}
\begin{enumerate}[noitemsep]
	\item Zucker im Wasser auflösen lassen
	\item Wasser in einem Topf zum Kochen bringen
	\item in einer großen Schüssel Kakaopulver im Tapiokamehl auflösen lassen
	\item Sobald das Wasser kocht: Schrittweise eine halbe Suppenkelle Wasser zum Teig geben und unterrühren (ggf. kneten)
	\item Solange kneten, bis der Teig zu einer glatten, zähen Masse wird (Teig sollte nicht mehr kleben)
	\item Teig in 4 Stücke teilen (nicht verwendete Teige in Frischhaltefolie packen)
	\item Den zu verarbeitenden Teig dünn ausrollen (ca. 1cm Durchmesser) und in Stücke schneiden
	\item Die abgeschnittenen Stücke in Kugeln formen
	\item Sobald alle Kugeln fertig geformt wurden: die gewünschte Anzahl an Kugeln entnehmen
	\item Den Rest in Tapiokamehl bedecken und im Kühlfach lagern
	\item Tapiokaperlen in kochendes Wasser geben
	\item Sobald sie an der Oberfläche schwimmen, sind die Tapiokas fertig 
	\item Topf ausschalten und die Tapiokaperlen für 5 Minuten im Wasser lassen
	\item Tapiokaperlen in kaltes Wasser geben
	\item Sobald die Tapiokaperlen abgekühlt sind: in das Zuckerwasser geben
\end{enumerate}