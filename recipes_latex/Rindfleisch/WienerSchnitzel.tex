\subsection{Wiener Schnitzel (2 Portionen)}
\paragraph{Zutaten}
\begin{itemize}[noitemsep]
	\item 2 Kalbsschnitzel Oberschale
	\item Salz
	\item Pfeffer
	\item 3 Eier
	\item Paniermehl
	\item Mehl
	\item 1 Flocke Butter
\end{itemize}
\paragraph{Zubereitung}
\begin{enumerate}[noitemsep]
	\item Mehl für das Panieren auf einen flachen Teller geben
	\item Eier in einen vertieften Teller tun und "anschlagen" (nicht komplett) 
	\item Paniermehl auf einen flachen Teller tun 
	\item Oberseite des Kalbsschnitzel mit bisschen Wasser befeuchten und mit dem Fleischhammer gleichmäßig platthauen
	\begin{description}[noitemsep, nolistsep]
		\item $\rightarrow$ einmal zwischendrin umdrehen 
	\end{description}
	\item Danach: beide Seiten mit Salz und Pfeffer würzen
	\item Fleisch im Mehl wenden, im Ei wenden und dann mit dem Paniermehl ummanteln 
	\item Öl in einer Pfanne heiß machen und das Schnitzel darin backen bis die Panade knusprig goldbraun ist
	\begin{description}[noitemsep]
		\item $\rightarrow$ regelmäßig die Pfanne schwenken
	\end{description}
	\item Danach: Butter in das Öl geben und eine Minute weiterbacken 
	\item Schnitzel aus der Pfanne nehmen und das überschüssige Öl mit Zewa wegmachen
\end{enumerate}
\begin{figure}[h]
	\centering
	\includegraphics[width= 0.75\textwidth]{Bilder/WienerSchnitzel}
\end{figure}
