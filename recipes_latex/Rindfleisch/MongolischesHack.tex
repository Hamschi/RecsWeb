\newpage
\titlelink{Mongolisches Hack (2 Portionen) }{https://www.youtube.com/watch?v=ImEDbyWuswU&list=PLwXqhx9ZM-a_VsqCvTGJoXphf10mSI_Jf&index=160&ab_channel=AaronandClaire}
\paragraph{Zutaten}
\begin{itemize}[noitemsep]
	\item 5 Zehen Knoblauch
	\item 5 Frühlingszwiebeln
	\item 5 Vietnamese Chilis (getrocknet, optional)
	\item 10g Ingwer
	\item 500g Rinderhack
	\item 3 EL Sojasoße (hell)
	\item 2 EL Zucker
	\item 1 EL Austernsoße
	\item 1 TL Hühnerbrühepulver
	\item 1/2 TL Sojasoße (dunkel)
	\item 1 EL Light Corn Sirup (Alternativ: Honig, Agavendicksaft, Ahornsirup,...)
	\item 2 EL Mirin
	\item Pfeffer (schwarz)
	\item 5 EL Wasser
	\item 1 EL Maisstärke
	\item 1/2 EL Sesamöl
	\item 1 Chili
\end{itemize}

\paragraph{Zubereitung}
\begin{enumerate}[noitemsep]
	\item Chili in dünne Scheibchen schneiden
	\item Weißen Teil der Zwiebel in dünne Scheiben schneiden, Rest in lange Röhren (ca. 5cm) schneiden
	\item Knoblauch fein schneiden, Ingwer reiben
	\item Soße für das Fleisch mischen: Sojasoßen, Zucker, Austernsoße, Hühnerbrühepulver, Mirin, Pfeffer, Wasser, Maisstärke
	\item Hackfleisch scharf anbraten (ca. 7-8 Minuten) bis es braun/dunkelbraun wird
	\item Hackfleisch zur Seite legen
	\item In der selben Pfanne Frühlingszwiebel, Knoblauch und die getrockneten Chili darin anbraten (ca. 1 Minute)
	\item Die dicken Teile der Frühlingszwiebel sowie die getrockneten Chilis mitbraten (ca. 30s)
	\item Soße dazugeben und anbraten bis die Soße dicker wird (ca. 1 Minute)
	\item Hack hinzugeben und alles ca. eine Minute anbraten
	\item Herd ausmachen und die grünen Zwiebeln unterrühren
	\item Auf Reis servieren und mit Chili garnieren
\end{enumerate}
\placeimage{Bilder/MongolianBeef}{11.7cm}{-4.8cm}{0.5}