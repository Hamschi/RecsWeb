\newpage
\subsection{Okonomiyaki (1 Portion)}
\paragraph{Zutaten}
\begin{itemize}[noitemsep]
	\item 200g Kohl
	\item 150g Schweinebauchstreifen (dünn)
	\item 100g Mehl
	\item 1 Pck. Dashipulver 
	\item 100ml Wasser
	\item 1 Ei
	\vspace{0.5cm}
	\item 1 EL Worcestershire Sauce
	\item 2 EL Ketchup
	\item 1 TL Sojasoße
	\item 2 EL Zucker
	\vspace{0.5cm}
	\item Mayonnaise 
	\item Aonori
	\item Bonito Flocken
\end{itemize}
\paragraph{Zubereitung}
\begin{enumerate}[noitemsep]
	\item Kohl in grobe Stücke schneiden und in eine Schüssel legen
	\item Dashipulver ins Mehl unterrühren
	\item Wasser unterrühren
	\item Ei unterrühren
	\item Teig in den Kohl unterrühren
	\item Die ganze Mischung in eine heiße Pfanne mit Öl schütten und die Fleischstreifen drauf legen 
	\item Deckel auf die Pfanne legen 
	\item Sobald die Farbe des Fleisches sich ändert: Okonomiyaki wenden (ca. 3 Minuten)
	\item Deckel auf die Pfanne legen und alles für 5 Minuten bei geringer Hitze braten
	\item Danach: auf einem Teller servieren
	\item Worcestersoße, Ketchup, Sojasoße und Zucker in einer Schüssel verrühren
	\item Soße auf dem Pfannkuchen verteilen
	\item Mayonnaise hinzufügen und mit einem Zahnstocher schön verteilen
	\item Aonori und Bonitoflocken darauf verteilen
\end{enumerate}
\placeimage{Bilder/Okonomiyaki}{11.5cm}{-5cm}{0.6}