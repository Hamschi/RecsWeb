\newpage
\subsection{Mamas Bánh Bao (24 Stk oder 18?) (not done yet)}
\paragraph{Zutaten}
\begin{itemize}[noitemsep]
	\item 900-1000g Mehl
	\item 3 Eier
	\item 200g Zucker
	\item 1 Pck. Vanillezucker
	\item 2 pck. Trockenhefe
	\item 450ml Milch
	\item 1 Prise Salz
	\item 300g Schweinenacken
	\item Fischsoße
	\item Pfeffer
	\item Backpulver
	\item MSG
	\item Öl
	\item 6 kleine/ 3 große Judasohr
	\item 1 Portion Glasnudeln
	\item 1 Zwiebel
	\item 1 Frühlingszwiebel
	\item 1 Karotte
	\item 3 Eier
\end{itemize}


\paragraph{Zubereitung}
\begin{enumerate}[noitemsep]
	\item Eiweiß aufschlagen
	\item Zucker, Salz und Vanillezucker mit dem Mixer unterrühren
	\item Trockenhefe in der Milch auflösen lassen 
	\item Milch und Eiweiß miteinander vermengen
	\item Mehl und Backpulver hinzufügen und vermengen
	\item Sobald der Teig zu fest wird: Hand zum Kneten benutzen
	\item Zitronensaft unterrühren
	\item Teig an einem warmen Ort 90 Minuten ruhen lassen
	\item Fleisch mit Fischsoße, Pfeffer, Backpulver, MSG und Öl marinieren
	\item Fleisch in einen Zerhacker geben
	\item Glasnudeln und Pilze in Wasser weich werden lassen
	\item Zwiebel, Karotte und Frühlingszwiebel klein schneiden und mit Eigelb zum Fleisch geben und vermengen
	\item Eier aufkochen
	\item Judas Ohr klein schneiden (ggf. harten Teil skippen)
	\item Glasnudeln auf 1.5cm runter kürzen und mit dem Rest vermengen
	\item Gekochte Eier in je 6 Spalten schneiden
	\item Sobald der Teig gut aufgegangen ist: eine Handvoll Teig nehmen und zu einem flachen Teig durchkneten/ ziehen
	\item Füllung mit einer Eierspalte auf den Teig legen, zumachen und dann dämpfen
	\item Nach x Minuten: raus nehmen und essen
\end{enumerate}