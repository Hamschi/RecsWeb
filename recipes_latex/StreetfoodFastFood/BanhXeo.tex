\newpage
\selectlanguage{vietnamese}
\subsection{Bánh Xèo (need pic)}
\paragraph{Zutaten}
\begin{itemize}[noitemsep]
	\item 400g Schweinebauch
	\item 100g Garnelen
	\item 3 EL Fischsoße
	\item 1 Prise Pfeffer
	\item 3 Frühlingszwiebeln (nur die Halme)
	\item 250 Reismehl (bột gạo)
	\item 50g Mehl
	\item 1/2 TL Kurkuma
	\item 250ml Sprudelwasser
	\item 200ml Kokosmilch
	\item 1 TL Salz
	\item 1/2 Gemüsezwiebel (alternativ eine gelbe Zwiebeln)
	\item 400g Sojasprossen
	\item 150g Đỗ
\selectlanguage{ngerman}
\end{itemize}
\paragraph{Zubereitung}
\begin{enumerate}[noitemsep]
	\item Schweinebauch und Garnelen kleinschneiden und anschließend mit Fischsoße und Pfeffer marinieren
	\item die Halme der Frühlingszwiebel in dünne Scheiben schneiden
	\item Reismehl und das Mehl in einer Schüssel miteinander vermengen
	\item das Sprudelwasser, Kokosmilch, Kurkuma und Salz in einem Messbecher mischen und die Zutaten darin sich auflösen lassen
	\item die Halme der Frühlingszwiebel in kleine Scheiben schneiden und in den Teig geben
	\item den Teig nun 1-3h ziehen lassen
	\vspace{0.5cm}
	\selectlanguage{vietnamese}
	\item Öl in der Pfanne heiß machen
	\item Sobald das Öl heiß ist: Herd auf mittel-heiß stellen
	\item mit einer Suppenkelle den Teig nochmal verrühren und eine Suppenkelle Teig gleichmäßig in der Pfanne verteilen
	\item Đỗ, Fleisch, Sojasprossen und und Zwiebeln auf die eine Seite des Pfannkuchens legen und einen Deckel auf die Pfanne legen
	\item nach 1.5-2 Minuten den Pfannkuchen zuklappen 
	\item solange alle 1-2 Minuten wenden wiederholen, bis er fertig zu sein scheint
\end{enumerate}