\section{Vorstellung}
In diesem Buch, welches kein Buch ist, sondern lediglich viele einzelne Rezepte beinhaltet, die ausgedruckt worden sind und in einem Ordner aufbewahrt werden, habe ich viele Rezepte gesammelt, die toll schmecken. Zumindest schmecken die meisten Rezepte toll. Das denke ich zumindest. Kann mich auch irren

\newpage
\section{Die Goldenen Regeln}
\begin{enumerate}[label=\Roman*]
	\item Vor dem Kochen mit Fleisch muss dieses zuerst sauber gemacht werden
	\item Vor dem Kochen mit Gemüse oder Obst muss dieses gewaschen werden
	\item Vor dem Kochen mit Reis muss dieser gründlich gewaschen werden
	\item Beim Kochvorgang muss regelmäßig abgeschmeckt werden
	\item Bei Frühlingszwiebeln und grünem Spargel wird das Ende entsorgt
	\item Beim Backen von Teig kommt zuerst	 die Butter, danach der Zucker gefolgt von dem Ei
	\item Beim Frittiervorgang werden die Teile, die noch nicht so weit sind, mit denen in der Mitte getauscht
	\item Beim Braten von Gemüse wird mit Wasser nachgegossen, wenn Flüssigkeit fehlen sollte
	\item Ober-/Unterhitze bei Blechkuchen und Käsekuchen
	\item Umluft bei allen Gerichten, die knusprig werden sollen
	\item Teig ruhen lassen: Backofen unvorgeheizt 5 Minuten auf 50$^\circ$C stellen für xy Stunden
	\item Rindfleisch immer gegen den Strich schneiden
	\item Wenn man mit keiner Fritteuse frittiert, dann legt man Zeitung auf dem Boden aus
	\item Das Öl wird mit der Herdstufe 9 erhitzt, beim Frittieren jedoch auf Stufe 6 oder 7 reduziert
	\item beim Herausnehmen des frittierten Essens muss man das Öl abtropfen lassen
	\item so gut es geht Pfeffer aus der Mühle benutzen!
\end{enumerate}

\newpage
\section{Grundlagen}
\subsection{Umrechnungstabelle}
\begin{tabular}{l | l | l | l | l}
1 EL & Butter & $\rightarrow$ & 15g & Butter \\
\hline
1 TL & Butter & $\rightarrow$ & 5g & Butter \\
\hline
1 EL & Mehl & $\rightarrow$ & 10g & Mehl \\
\hline
1 TL & Mehl & $\rightarrow$ & 3g & Mehl \\
\hline
1 cup & Mehl & $\rightarrow$ & 120g & Mehl \\
\hline
1 EL & Milch &  $\rightarrow$ & 15ml & Milch \\
\hline
1 TL & Milch & $\rightarrow$ & 5ml & Milch \\
\hline
1 cup & Milch & $\rightarrow$ & 240ml & Milch \\
\hline
1 EL & Wasser &  $\rightarrow$ & 15ml & Wasser \\
\hline
1 TL & Wasser & $\rightarrow$ & 5ml & Wasser \\
\hline 
1 cup & Wasser & $\rightarrow$ & 240ml & Wasser \\
\hline 
1 EL & Öl & $\rightarrow$ & 10g & Öl \\
\hline
1 EL & Salz & $\rightarrow$ & 20g & Salz \\
\hline
1 TL & Salz & $\rightarrow$ & 5g & Salz \\
\hline
1 EL & Zucker & $\rightarrow$ & 15g & Zucker \\
\hline
1 TL & Zucker & $\rightarrow$ & 5g & Zucker \\
\hline
1 cup & Zucker & $\rightarrow$ & 225g & Zucker \\
\hline
60g & Petersilie & $\rightarrow$ & 4 EL & Petersilie \\
\hline 
1/2 cup & Petersilie & $\rightarrow$ & 30g & Petersilie \\
\end{tabular}

\subsection{Schweinefleisch sauber machen}
\begin{enumerate}[noitemsep]
	\item Fleisch wird kurz unter das Wasser gehalten
	\item ggf. Fleisch schneiden
	\item Einen Topf mit Wasser erhitzen und Salz darin auflösen lassen
	\item Sobald das Wasser brodelt: Topf vom Herd nehmen und einen Schuss Essig rein
	\item Fleisch für 2-3 Minuten rein und dann mit einem Sieb abtropfen
\end{enumerate}

\subsection{Zimt \& Zucker: Verhältnis}
Das Verhältnis ist 1:10 (Zimt 1 und Zucker 10)