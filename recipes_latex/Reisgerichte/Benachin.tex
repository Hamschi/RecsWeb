\newpage
\subsection{Benachin/ Jollof Reis (8 Portionen)}
\paragraph{Zutaten}
\begin{itemize}[noitemsep]
	\item 1 Tomate (geviertelt und entkernt)
	\item 1 Paprika (grob geschnitten)
	\item 1 Habanero 
	\item 1 Chili
	\item 4 Zehen Knoblauch (geschält)
	\item 1 EL Ingwer (fein)
	\item 470ml Wasser
	\vspace{0.5cm}
	\item 80ml Olivenöl
	\item 1 Zwiebel (rot)
	\item 1 TL Salz
	\item 3-4 EL Tomatenmark 
	\item 2 EL Paprikapulver
	\item 1 TL Currypulver
	\item 1 TL Kreuzkümmel
	\item 1 TL Thymian (getrocknet)
	\item 1/2 TL Pfeffer (schwarz)
	\item 1/4 TL Kurkuma
	\item 1 Lorbeerblatt
	\item 1 EL Hühner Bouillon Paste
	\item 2 Halme Frühlingszwiebel (geschnitten) (optional)
	\item 1/4 Bund Koriander (fein gehackt) (optional)
\end{itemize}
\paragraph{Zubereitung}
\begin{enumerate}[noitemsep]
	\item Tomate, Paprika, Habanero, Chili, Knoblauch, Ingwer und Wasser in einem Mixer pürieren bis eine glatte Masse entsteht
	\item Zwiebeln im Öl salzen und glasig anschwitzen
	\item Platz in der Mitte schaffen und den Tomatenmark für 2-3 Minuten anbraten, dabei das Tomatenmark kontinuierlich umrühren 
	\item Gewürze in den Topf geben und alles miteinander vermengen
	\item Reiskörner zur Tomatenpaste hinzufügen und vermengen, sodass jeder Korn von der Tomatenpaste bedeckt ist
	\item Hühnerbouillon Paste und Lorbeerblatt hinzufügen
	\item Tomatensoße rein schütten und zum Kochen bringen
	\item Sobald die Soße kocht: Deckel auf den Topf und Herd auf mittlere Flamme senken und 20 Minuten köcheln lassen
	\item Danach: Reis für 12 Minuten ruhen lassen (Deckel auf dem Topf lassen)
	\item Danach: alles mit einer Gabel auflockern und dann mit Frühlingszwiebeln und Koriander verzieren
\end{enumerate}
\placeimage{Bilder/Benachin}{11.5cm}{-5.5cm}{0.6}
