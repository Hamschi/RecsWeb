\newpage
\subsection{Omurice (3 Portionen) (not done yet)}
\paragraph{Zutaten}
\begin{itemize}[noitemsep]
	\item 100g Hähnchenbrust
	\item 1/4 Karotte
	\item 1 Shiitake Pilz
	\item 160g Reis pro Person (gekocht)
	\item 2 EL Ketchup
	\item Salz
	\item Pfeffer
	\item Olivenöl zum Braten
	\vspace{0.5cm}
	\item 3x 3 Eier (groß)
	\item 3x 1 EL Wasser
	\item 3x 1 Prise Salz
	\vspace{0.5cm}
	\item 3x 1 EL Wasser
	\item 3x 1 EL Ketchup
	\item 3x 1 EL Worcestershire Soße
	\item 3x 1 EL Butter
\end{itemize}
\paragraph{Zubereitung}
\begin{enumerate}[noitemsep]
	\item Hähnchen in kleine Würfel schneiden (ca. 1cm groß)
	\item Zwiebel in kleine Würfel schneiden
	\item Stiel vom Pilz entfernen und den Pilz in kleine Stücke schneiden 
	\begin{description}[noitemsep, nolistsep]
		\item $\rightarrow$ Gemüse sollten alle ungefähr die gleiche Größe haben 
	\end{description}		
	\item Hähnchen für paar Sekunden in Olivenöl anbraten und das Gemüse und den Reis hinzufügen 
	\item Ketchup und Salz hinzufügen und auf mittlerer-hoher Hitze für 3-4 Minuten braten
	\vspace{0.5cm}
	\item Wasser, Ketchup, Worcestershire Soße in einer Pfanne vermengen (mittlere Hitze) 
	\item Hitze erhöhen und anschließend Butter hinzufügen
	\begin{description}[noitemsep, nolistsep]
		\item $\rightarrow$ Sobald die Butter schmilzt, ist die Soße fertig
	\end{description}
	\item Eier, Wasser und Salz in einer Schüssel grob schlagen und durch ein Sieb gehen lassen
	\item Omelette in (eher kleinen) Pfanne in Olivenöl braten 
\end{enumerate}