\newpage
\subsection{Oyako-Don (2 Portionen)}
\paragraph{Zutaten}
\begin{itemize}[noitemsep]
	\item Reis
	\item 400g Hähnchenoberkeule
	\item 1 Gemüsezwiebel
	\item 6 Eier (evtl. auch mehr)
	\item 2 EL Zucker
	\item 4 EL Sojasoße (hell)
	\item 6 EL Mirin
	\item 200ml Dashi
	\item Mitsuba (oder irgendwas Grünes)
\end{itemize}
\paragraph{Zubereitung}
\begin{enumerate}[noitemsep]
	\item den Reis kochen
	\item die Zwiebeln in Spalten (Wedges) schneiden 
	\item das Fleisch in mundgerechte Stücke schneiden
	\item die Eier in Eier in einer Schüssel schlagen
	\item Dashi, Zucker, Sojasoße, Mirin und die Zwiebeln in einem Topf zum Köcheln bringen
	\item Sobald es köchelt: Fleisch dazugeben und solange auf mittlerer Hitze kochen bis das Fleisch durch ist 
	\item Danach: den fertig gekochten Reis in eine Schüssel geben
	\item die Hälfte vom Fleisch ins eine Pfanne geben und erhitzen
	\item Sobald die Flüssigkeit kocht: die Hälfte vom Ei gleichmäßig, kreiselnd und langsam in die Pfanne geben 
	\item Sobald das Ei drin ist: mit dem Kochlöffel das Ei flach-rund Formen 
	\item Sobald das Fleisch und das Ei eine Masse bilden: die Hälfte der Hälfte des Eis auf ähnliche Weise hinzugeben
	\item den Herd ausschalten, den Deckel drauflegen und 10 Sekunden warten 
	\item das Ei auf den Reis verlegen 
\end{enumerate}
\begin{figure}[h]
\centering
\includegraphics[width=.62\textwidth]{Bilder/Oyakodon}
\end{figure}