\newpage
\titlelink{Japanese Curry (3-4 Portionen)}{https://www.youtube.com/watch?v=u5U-FH7o7v8&list=PLwXqhx9ZM-a_VsqCvTGJoXphf10mSI_Jf&index=163&ab_channel=Sudachi\%7CJapaneseRecipes}
\paragraph{Zutaten}
\begin{itemize}[noitemsep]
	\item 300g Zwiebeln
	\item 1 EL Olivenöl
	\item 1 Prise Salz
	\item Wasser
	\item 300g Rindfleisch
	\item 1 EL Butter
	\item 1 Prise Salz
	\item 2 Zehen Knoblauch
	\item 150g Karotten (geschält)
	\item 200g Kartoffeln (geschält, festkochend)
	\item 100g jap. Curry-Mehlschwitze
	\item 90\% Wasser entsprechend der Roux-Packungsanweisung
	\item 10\% Wein entsprechend Wasser der Roux-Packung
	\item 2 TL Kaffeepulver (optional)
\end{itemize}

\paragraph{Zubereitung}
\begin{enumerate}[noitemsep]
	\item Zwiebeln in Scheiben schneiden
	\item Zwiebeln im Olivenöl braten bis sie karamellisieren
	\item Nach ca. 10 Minuten: Herdstufe senken und Salz und Wasser hinzufügen
	\item 30 Minuten lang braten
	\begin{description}[noitemsep, nolistsep]
		\item $\rightarrow$ hin und wieder umrühren, damit nichts kleben bleibt 
		\item $\rightarrow$ Immer 1-2 EL Wasser dazugeben, wenn die Zwiebeln kleben bleiben sollten
	\end{description}
	\item Fertig, sobald die Zwiebeln eine dunkle, Sojasoßen-Farbe angenommen haben
	\item Während dem Zwiebelprozess: Fleisch in mundgerechte Stücke schneiden, Kartoffeln etwas größer, Karotten grob schneiden, Knoblauch hacken
	\item Danach: Butter in der Pfanne warm schmelzen und den Knoblauch anbraten
	\item Fleisch hinzufügen und salzen
	\item Sobald das Fleisch außen braun wird: Kartoffeln und Karotten (ca. 2 Minuten) unterrühren
	\item Die Zwiebeln gleichmäßig unterrühren
	\item Mit Wein ablöschen und anschließend das Wasser dazugeben
	\item Curry zum Kochen bringen
	\item Sobald es kocht: Herdstufe senken und mit einem Topfdeckel 20 Minuten köcheln lassen
	\begin{description}[noitemsep, nolistsep]
		\item $\rightarrow$ Hin und wieder nachschauen, ob Schaum zu sehen ist und entfernen
	\end{description}
	\item Danach: Curry-Mehlschiwtze im Curry auflösen und ohne Deckel 5-10 Minuten köcheln lassen
	\item Außerdem: Kaffeepulver unterrühren (optional)
	\item Mit Reis servieren
\end{enumerate}
\placeimage{Bilder/japaneseCurry}{11.5cm}{-4cm}{0.6}