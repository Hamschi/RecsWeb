\newpage
\titlelink{Kartoffelgratin (3-4 Portionen)}{https://youtu.be/gs429aLCMdE?si=ODSc3TYB7iYkKvep}
\paragraph{Zutaten}
\begin{itemize}[noitemsep]
	\item 500ml Sahne
	\item 1.5 Kartoffeln (festkochend)
	\item 2 Zweige Rosmarin
	\item 2 Zehen Knoblauch
	\item Salz
	\item Pfeffer
	\item Muskatnuss
	\item Butter für die Auflaufform
	\item 50g Parmesan
\end{itemize}
\paragraph{Zubereitung}
\begin{enumerate}[noitemsep]
	\item Kartoffeln schälen
	\item Knoblauchzehen mit dem Messerrücken anpressen
	\item Sahne, Knoblauchzehen, Rosmarin, Salz, Pfeffer und Muskatnuss in einem Topf sammeln 
	\item Sahne aufkochen lassen
	\begin{description}[noitemsep, nolistsep]
		\item \rarrow Abschmecken! Soll salzig sein! Und gut schmecken!
	\end{description}
	\item Kartoffeln in dünne, gleichmäßige Scheiben schneiden
	\item Auflaufform mit Butter einreiben
	\item Eine Schicht Kartoffeln in die Auflaufform geben
	\item Etwas von der Sahne über diese Schicht geben
	\item Zwischendrin salzen und Pfeffern (optional)
	\item Eine Schicht Parmesan darüber reiben
	\item Die letzten vier Schritte wiederholen bis die die Zutaten aufgebraucht sind
	\item Auflaufform abdecken (z.B. mit Alufolie) und dann in den Ofen geben
	\item Ofen auf 180\textdegree C schalten 
	\item Sobald die Sahne kocht (ca. 45 Minuten): Bedeckung entfernen und weiterbacken bis der Gratin goldbraun ist 
	\item Für weitere 30 Minuten im Ofen lassen
\end{enumerate}
\placegraphic{Bilder/Kartoffelgratin}{0.5}