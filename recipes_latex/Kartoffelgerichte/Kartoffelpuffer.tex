\newpage
\titlelink{Kartoffelpuffer/Reibekuchen (2 Portionen)}{https://youtu.be/rODf39kwEys?si=6pwrWkETfm_sHNcQ}
\paragraph{Zutaten}
\begin{itemize}[noitemsep]
	\item 500g Kartoffeln (festkochend) (ca. 5-7 Stück)
	\item 1/2 Zwiebel
	\item Muskat
	\item Salz
	\item Pfeffer
	\item etwas Rapsöl
\end{itemize}

\paragraph{Zubereitung}
\begin{enumerate}[noitemsep]
	\item Kartoffeln und Zwiebeln mit einer Grobreibe in eine Schüssel reiben 
	\item Die geriebene Masse in einen Sieb geben und das Wasser daraus extrahieren und in einem Behälter sammeln 
	\begin{description}[noitemsep]
		\item $\rightarrow$ Schüssel stehen lassen, bloß nicht bewegen!
	\end{description}
	\item Masse zurück in die Schüssel geben 
	\item Ei, Salz, Pfeffer und Muskat in die Masse geben und alles vermengen
	\item Stärke hat sich von dem Kartoffelwasser getrennt: Wasser wegkippen
	\item Kartoffelmasse zur Stärke geben und gut verrühren
	\item Öl in einer Pfanne erhitzen und die Kartoffelpuffer portionsweise in die Pfanne geben
	\item Sobald sie drin sind: Hitze sofort senken, damit die Kartoffelpuffer langsam und gleichmäßig gebraten werden bis sie eine Kruste bekommen (ca. 5 Minuten)
	\item Sobald die Kartoffelpuffer eine knusprige Farbe haben: wenden 
	\item Sobald Kartoffelpuffer fertig sind: auf einen Behälter mit Küchenpapier legen, um Öl aufzusaugen
	\item 
\end{enumerate}
\placegraphic{Bilder/Kartoffelpuffer}{0.7}