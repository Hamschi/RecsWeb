\newpage
\subsection{Kartoffelpüree (4-5 Portionen)}
\paragraph{Zutaten}
\begin{itemize}[noitemsep]
	\item 1kg Kartoffeln (mehligkochend)
	\item 1l Wasser
	\item 1 EL Salz
	\item 1 Schuss Milch (lauwarm)
	\item 2 EL Butter (alternativ Margarine)
	\item 1 Prise Muskatnuss
	\item 1 Prise Salz
\end{itemize}
\paragraph{Zubereitung}
\begin{enumerate}[noitemsep]
	\item Kartoffeln schälen und in etwa gleich große Stücke schneiden
	\item Kartoffeln in einem Topf mit Wasser geben
	\begin{description}[noitemsep, nolistsep]
		\item $\rightarrow$ so viel Wasser, dass die Kartoffeln gerade so bedeckt sind
	\end{description}
	\item Wasser salzen, aufkochen und bei geschlossenem Deckel ca. 20 Minuten köcheln lassen
	\begin{description}[noitemsep, nolistsep]
		\item $\rightarrow$ darauf achten, dass sich die Kartoffeln nicht am Topf anlegen
	\end{description}
	\item den Topf vom Herd nehmen und das Wasser abgießen
	\item Die Kartoffeln zerstampfen und die Milch zugießen
	\item Butter, Salz und Muskatnuss hinzufügen 
	\item nochmals mit dem Kartoffelstampfer zerstampfen und nur ganz kurz mit einem Schneebesen flaumig schlagen, bis ein richtiger Kartoffelbrei entsteht
	\item Tipp 1: für eine deftigere Variante kann man auch geröstete Zwiebel- und Speckstücke in das Püree einarbeiten
\end{enumerate}
\begin{figure}[h]
\centering
\includegraphics[width=0.9\textwidth]{Bilder/Kartoffelpueree}
\end{figure}