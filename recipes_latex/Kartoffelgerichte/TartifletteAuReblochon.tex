\newpage
\subsection{Tartiflette au Reblochon (not done yet))}
\paragraph{Zutaten}
\begin{itemize}[noitemsep]
	\item 300-400g Kartoffeln (ca. 4 mittelgroße)
	\item 1 Zwiebel
	\item 80g-100g Schinkenspeck
	\item 1-2 EL Butter
	\item 1 Becher Creme Fraiche
	\item 225g Reblochon 
	\item Gewürze: Salz, Pfeffer, Muskat
	\vspace{0.5cm}
	\item 0,5-1 Salatkopf
	\item 1 EL Olivenöl
	\item 2 EL Essig oder Balsamico
	\item 1 EL Senf Ancienne
	\item Salz, Pfeffer
	\item 3 EL Walnusskerne (grob gehackt)
	\item ggf. Tomate und Gurke	
	\vspace{0.5cm}
	\item 50 – 150 g gemischte Wurst: gekochter und roher Schinken, verschiedene Salami, etc.
\end{itemize}
\paragraph{Zubereitung}
\begin{enumerate}[noitemsep]
	\item Kartoffeln zu Pellkartoffeln verarbeiten
	\item Pellkartoffeln, Zwiebeln und Speck grob schneiden
	\item Butter in einer Pfanne schmelzen lassen und Zwiebeln und Speck darin braten 
	\item Creme Fraiche dazumischen und mit Gewürzen abschmecken
	\item Kartoffelstücke hinzugeben und alles mischen 
	\item Masse in 2-3 kleinere Auflaufformen geben 
	\item Reblochon in Scheiben schneiden und dick belegen 
	\item Salat in Mundgerechte Stücke schneiden 
	\item Öl, Senf und Balsamico mischen und mit Salz und Pfeffer abschmecken
	\item Walnusskerne darüber streuen und ggf. Tomaten und Gurkenscheiben hinzufügen
	\item Salami und Schinken in feine Scheiben schneiden und auf einem Servierbrett oder Teller schön anrichten
	\item Auflauf bei 180 \textdegree C (Umluft) für 20 Minuten in den Ofen legen
	\begin{description}[noitemsep, nolistsep]
		\item $\rightarrow$ fertig, sobald Käse geschmolzen ist und eine goldbraune Farbe annimmt
	\end{description}
	\item mit Cornichons, Wurst, Baguette und Salat servieren	
\end{enumerate}