\clearpage
\titlelink{Chicken Biryani (not done yet)}{https://youtu.be/PmqdA05OXuI?si=apncI38GESP7S1Sh}

\paragraph{Zutaten}
\begin{itemize}[noitemsep]
	\item 300g Basmatireis
	\item 2l Wasser
	\item Salz 
	\item 2 EL Ghee
	\item 1 TL Öl
	\item 1 Lorbeerblatt
	\item 2 Kardamomkapseln (grün)
	\item 1 Kardamomkapsel (schwarz)
	\item 2.5cm Zimt
	\item 8 Körner Pfeffer (schwarz)
	\item 4 Nelken
	\vspace{0.5cm}
	\item 3 EL Öl
	\item 2 TL Ghee
	\item 2 Kartoffeln (geschält)
	\item 3 Kardamomkapseln (grün)
	\item 1 Kardamomkapsel (schwarz)
	\item 4 Nelken
	\item 2.5cm Zimt
	\item 1 Lobeerblatt
	\vspace{0.5cm}
	\item 600g Hähnchen
	\item Salz
	\item 1/2 TL Kurkuma
	\item 1 EL Chilipulver
	\item 280g Quark
	\item 1 EL Ingwer-Knoblauch-Paste
	\item 1 große Zwiebel (geröstet)
	\item 1/2 Cup Tomaten (gewürfelt)
	\item Wasser
	\item 2 Eier (gekocht)
	\item Garnieren: Röstzwiebeln und Minzblätter
\end{itemize}

\paragraph{Zubereitung}
\begin{enumerate}[noitemsep]
	\item Ghee in einer Pfanne erhitzen, anschließend die Gewürze dazugeben und die Kartoffeln darin goldbraun braten
	\item Hähnchen mit Quark, Inger-Knoblauch-Paste, Salz, Kurkuma, Chilipulver, Zwiebeln und Tomaten marinieren
	\item Hähnchen in die Pfanne geben und Wasser dazugeben
	\item Pfanne mit einem Deckel abdecken und solange bei mittlerer Hitze köcheln lassen, bis es fertig ist
	\item In einer anderen Pfanne: Ghee und Öl erhitzen, anschließend die Gewürze und das Wasser und Salz hinzufügen und die Hitze reduzieren
	\item Den gewaschenen Reis hinzugeben und auf hoher Hitze kochen lassen bis es zu 75\% durch ist
	\item Die Eier und den Reis zum Hähnchen geben und die Pfanne mit einem Deckel bedecken
	\item Für 8 Minuten auf hoher Hitze kochen, anschließend 2 Minuten auf niedriger Hitze
	\item Mit Minzblättern und Röstzwiebeln garnieren
\end{enumerate}