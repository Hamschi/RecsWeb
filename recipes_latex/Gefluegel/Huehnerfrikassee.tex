\newpage
\subsection{Hühnerfrikassee}
\paragraph{Zutaten}
\begin{itemize}[noitemsep]
	\item 2 EL Butter
	\item 400g Hähnchenbrust
	\item 1 Prisen Salz
	\item 1,5 Zehen Knoblauch
	\item 1 Prise Pfeffer
	\item Schuss Sojasoße (hell) (optional) 
	\item 4 große Champignons (weiß)
	\item 1 Zwiebel (weiß)
	\item 100ml Weißwein
	\item 400ml Gemüsebrühe
	\item 200ml Sahne
	\item 1-2 Prisen Salz
	\item 1-2 EL Mehl
	\item 1-2 Spritzer Zitronensaft
	\item 2 EL Crème fraîche
	\item 1 TL Zitronenschale (optional)
\end{itemize}
\paragraph{Zubereitung}
\begin{enumerate}[noitemsep]
	\item Hähnchenbrust in feine Scheiben und Zwiebel in feine Würfel schneiden
	\item 1 EL Butter in der Pfanne bei mittlerer Hitze schmelzen lassen
	\item Hähnchen in die Pfanne geben und braten, bis es kaum rosa Stellen mehr gibt
	\item Gewürze zu dem Hähnchen geben 
	\item Während das Hähnchen brät, 1 EL Mehl hinzugeben 
	\item Nach dem Braten das Hähnchen zur Seite legen und Butter in der Pfanne schmelzen 
	\item Zwiebeln in der Butter blondieren und 1 EL Mehl dazugeben 
	\item Zwiebeln mit Weißwein ablöschen und und diese dann bei hoher Flamme rühren 
	\item Brühe in regelmäßigen Abständen dazugeben
	\item Währenddessen: Frühlingszwiebeln in Röhrchen schneiden
	\item Pilze auch kleinschneiden 
	\item geben Ende des Bratprozzesses: Sahne und Salz hinzugeben
	\item Nach Minuten: Pilze und Champignons hinzugeben 
	\item Nach 1 Minute: Hitze auf mittlere Stufe reduzieren und Fleisch in die Pfanne geben
	\item Crème fraîche und Zitronensaft unterrühren (auch ggf. Zitronenschale)
	\item 1-2 Minuten weiterbraten
\end{enumerate}
\placeimage{Bilder/Huehnerfrikassee}{10.8cm}{-4cm}{0.7}