\newpage
\subsection{Dakgangjeong - Korean Fried Chicken (not done yet)}
\paragraph{Zutaten}
\begin{itemize}[noitemsep]
	\item 350g Hähnchenfilet
	\item 3 Prisen Salz
	\item 1/2 TL Ingwerpulver
	\item 1 TL Knoblauch (gehackt)
	\item 3 EL Kochwein (alternativ: Milch)
	\item 1/2 TL Pfeffer (schwarz)
	\item 2 EL Zucker
	\item 1 EL Cayennepfeffer (engl. Chillli Pepper)
	\item 1 TL Knoblauch (gehackt)
	\item 1,5 EL Sojasoße (hell)
	\item 1/2 EL Austernsoße
	\item 1/2 EL Ketchup
	\item 3 EL Chili Pepper Oil 
	\item 3,5 EL Stärkesirup 
	\item 40g Bratmehl (20g Mehl, 20g Maisstärke, Salz)
	\item 60g Reismehl
	\item 1 Prise Salz
	\item 100ml Wasser  
	\item 1 Chilischote (gehackt) (optional)
	\item 2 EL Erdnüsse (gehackt)
\end{itemize}
\paragraph{Zubereitung}
\begin{enumerate}[noitemsep]
	\item das Fleisch in mundgerechte Stücke schneiden 
	\item Salz, Ingwerpulver, Kochwein und Pfeffer zum Fleisch geben und alles verkneten
	\item Fleisch zur Seite legen
	\item in einer Schüssel Zucker, Cayennepfeffer, Knoblauch, Sojasoße, Austernsoße, Ketchup, Chillipefferöl und das Stärkesirup gut vermischen
	\item mit Bratmehl, Reismehl, Salz und Wasser einen Teig formen
	\item das Fleisch zum Teig geben und den Teig gleichmäßig verteilen
	\item Fleisch nun in heißem Öl frittieren 
	\item sobald das Fleisch goldbraun ist: entnehmen und 3 Minuten warten
	\item danach: weiterfrittieren bis das Fleisch dunkelbraun ist und entnehmen
	\item die Soße von eben in einer Pfanne erhitzen bis es anfängt zu kochen 
	\item Danach: 30s kochen lassen und dabei kontinuierlich rühren
	\item Danach: Fleisch und Chilischote hinzugeben und gleichmäßig mischen
	\item Erdnüsse hinzugeben und fertig
\end{enumerate}