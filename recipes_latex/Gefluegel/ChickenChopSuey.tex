\clearpage
\titlelink{Chicken Chop Suey (2 Portionen) (not done yet)}{https://youtu.be/ebc6PG9_oNs?si=lPn5uf6q0WB2iQ3-}

\paragraph{Zutaten}
\begin{itemize}[noitemsep]
	\item 180g Hähnchenbrust
	\item 1/2 TL Tapiokamehl (alternativ: Maisstärke)
	\item 1 Prise Salz
	\item 1 EL Wasser
	\item 1 Tropfen Öl
	\vspace{0.5cm}
	\item 1 EL Sojasoße
	\item 1/2 EL Sojasoße (dunkel)
	\item 2 EL Austernsoße 
	\item 1/2 TL Sesamöl
	\item 1 EL Tapiokamehl (alternativ: Maisstärke)
	\item 1 Prise Salz
	\item 1/4 TL Pfeffer (weiß)
	\item 4 EL Wasser
	\vspace{0.5cm}
	\item 1/2 Zwiebel
	\item 2 Zehen Knoblauch
	\item 1 Karotte (klein)
	\item 1 Handvoll Zuckererbse
	\item 3 Babymais
	\item 3 Champignon
	\item 1 Handvoll Mungbohnen
\end{itemize}

\paragraph{Zubereitung}
\begin{enumerate}[noitemsep]
	\item Karotten in dünne Scheiben schneiden, Champignon in Scheiben schneiden, Stämme aus den Zuckererbsen entfernen, Zwiebeln und Knoblauch fein schneiden und Baby-Mais schräg halbieren
	\item Hähnchen in dünne Scheiben schneiden (ca. 5mm dick)
	\item In einer Schüssel Tapiokamehl, Wasser und Hähnchen miteinander vermengen
	\item Einen Tropfen Öl hinzufügen, umrühren und zur Seite legen
	\item In einer Schüssel Sojasoßen, Austernsoße, Sesamöl, Tapiokamehl, Salz, Pfeffer und Salz zu einer Soße vermengen
	\item Öl in einem Wok erhitzen und das Hähnchen dazugeben
	\item Sobald das Fleisch nicht mehr rosa ist: sofort entnehmen
	\item Öl in die Pfanne geben und Zwiebeln und Knoblauch anschwitzen
	\item Karotten für ca. 15s im Wok braten
	\item Baby-Mais und Zuckererbsen dazugeben und für 20 weitere Sekunden braten
	\item Hähnchen wieder in die Pfanne geben inkl. Mungbohnen und Pilzen und für 30s braten
	\item Soße nochmals kurz anrühren und in die Pfanne geben und umrühren
	\item Sobald die Flüssigkeit dickflüssig wird: Alles auf einen Teller kippen
\end{enumerate}