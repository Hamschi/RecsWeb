\newpage
\subsection{Chicken Korma (3 Portionen)}
\paragraph{Zutaten}
\begin{itemize}[noitemsep]
	\item 5 Pflanzenöl zum Braten
	\item 2 Zwiebeln
	\item 10 Cashew Kerne
	\item 1 Cup Joghurt
	\item 1/2 TL Kurkuma
	\item 1 TL Korianderpulver
	\item 1 TL Chilipulver
	\item 2 TL Garam Masala
	\item 1 Stück Zimt
	\item 3 Kardamom (grün)
	\item 1 Kardamom (schwarz)
	\item 5 Nelken
	\item 10 Pfefferkörner
	\item 1 Lorbeerblatt
	\item 1 TL Ingwer 
	\item 1 Zehe Knoblauch
	\item 450g Hähnchenschenkel
	\item 1 Cup Wasser
	\item Salz
	\item Koriander zum Verzieren
\end{itemize}
\paragraph{Zubereitung}
\begin{enumerate}[noitemsep]
	\item Knoblauch und Ingwer feinhacken, Fleish in kleine Brocken schneiden
	\item Zwiebel kleinschneiden und im Öl braten bis sie dunkelbraun sind
	\item Zwiebeln und die Cashew Kerne in einem Mixer zu einer glatten Masse pürieren 
	\item Chilipulver, Kurkuma, Korianderpulver, Garam Masala zum Joghurt geben und verrühren
	\item In dem Topf, in dem man die Zwiebeln geröstet hat, den Zimt, Kardamom, Nelken, Pfeffer und das Lorbeerblatt anbraten bis die Nelken aufpoppen (wenige Sekunden)
	\item Danach Ingwer und Knoblauch dazugeben 
	\item Nach wenigen Sekunden das Fleisch darin braten bis es nicht mehr rosa ist
	\item Die Hitze auf niedrig-mittel reduzieren und den Joghurt sowie die Zwiebelpaste hinzufügen
	\item Alles gleichmäßig vermengen und mit Salz abschmecken
	\item Alles solange braten, bis das Öl am Rand der Pfanne zu sehen ist (ca. 10 Minuten)
	\item Anschließend das Wasser hinzufügen und für weitere 10-15 Minuten braten
	\item Mit Koriander verzieren
\end{enumerate}
\placeimage{Bilder/ChickenKorma}{10.5cm}{-5.5cm}{0.75}

