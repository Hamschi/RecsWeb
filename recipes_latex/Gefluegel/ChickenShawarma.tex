\newpage
\titlelink{Chicken Shawarma (2-4 Portionen)}{https://amateurprochef.com/2025/01/14/chicken-shawarma-super-easy/}

\paragraph{Zutaten}
\begin{itemize}[noitemsep]
	\item 900g Hähnchen (Brust oder Schenkel)
	\item 1 Tomate
	\item 1/2 Gurke
	\item Eingelegtes Gemüse
	\item Rote Zwiebeln
	\item Knoblauch Toum
	\item 2-4 Pitas
	\vspace{0.5cm}
	\item Salz
	\item 4 Zehen Knoblauch
	\item 1 TL Pfeffer
	\item 1 EL Knoblauchpulver
	\item 1 EL Zwiebelpulver
	\item 1 TL Kreuzkümmel
	\item 1 TL Kurkuma
	\item 1 TL rotes Chilipulver
	\item 2 EL Tomatenmark
	\item 1/2 Zitrone
	\item 3 EL Olivenöl
	\vspace{0.5cm}
	\item 1 Zwiebel (rot, groß)
	\item 1 EL Koriander (gehackt)
	\item 2 EL Zitronensaft
	\item 1 EL Sumac (Rhu)
\end{itemize}

\paragraph{Zubereitung}
\begin{enumerate}[noitemsep]
	\item Hähnchen in Streifen schneiden
	\item Hähnchen für 30 Minuten marinieren: Olivenöl, Tomatenmark, Zitronensaft von der Zitrone, Knoblauchpulver, Zwiebelpulver, Kreuzkümmel, Chilipulver, Kurkuma, Salz und Pfeffer  
	\item Hähnchen in einer Pfanne anbraten
	\item Währenddessen: Gemüse in Streifen schneiden
	\item Sobald das Fleisch durch ist: Pita auf das Fleisch legen, damit es durch wird und Fleischsaft aufnimmt
	\item Pita mit viel Toum bestreichen und anschließend mit Fleisch und Gemüse belegen
	\item Optional: Mit Zwiebeln, Koriander, Zitronensaft und/oder Sumac verfeinern
\end{enumerate}

\placeimage{Bilder/ChickenShawarma}{10.5cm}{-5.5cm}{0.65}