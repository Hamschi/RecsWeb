\titlelink{Eggnog (2 Portionen) (not done yet)}{https://youtu.be/ukB6v94vyKc?si=yGQX4GyjgYl_yHpt}

\paragraph{Zutaten}
\begin{itemize}[noitemsep]
	\item 710ml Vollmilch
	\item 6 Eigelbe
	\item 220g Zucker
	\item 1/2 Vanilleschote (alternativ: 2 TL Vanilleextrakt)
	\item 1 TL Zimt
	\item 1/4 TL Nelke (Pulver, gehäuft)
	\item 7g Ingwerpulver
	\item 1.5 TL Muskatnuss (gerieben)
	\item 60ml Rum (optional)
	\item 350ml Heavy Cream
\end{itemize}


\paragraph{Zubereitung}
\begin{enumerate}[noitemsep]
	\item Milch in einen großen Topf geben und heiß werden lassen, sodass es anfängt zum Dampfen
	\begin{description}[noitemsep, nolistsep]
		\item \rarrow nicht zum Kochen bringen 
	\end{description}
	\item Eigelbe in eine Schüssel geben mit einem Mixer schlagen
	\item Nachdem eine Masse entsteht: zum Schlagen parallel Zucker in kleinen Zeitabständen hinzugeben
	\begin{description}[noitemsep, nolistsep]
		\item \rarrow Immer nur paar EL-viel Zucker auf einmal rein schütten 
	\end{description}
	\item Sobald die Masse dickflüssig und cremig aussieht und die Milch dampft: Die Creme händisch umrühren und währenddessen in kleinen Zeitabständen 3/4 Kelle Milch dazugeben und unterrühren
	\item Diesen Schritt solange wiederholen, bis diese Mischung warm und flüssiger wird
	\item Diese Flüssigkeit in den Topf schütten mit dem Vanillemark
	\item Bei niedriger Hitze alles verrühren für 10-15 Minuten bis es eine schöne, dickflüssige Konsistenz bekommt 
	\begin{description}[noitemsep, nolistsep]
		\item \rarrow Nicht zum Kochen bringen. Ungefähr bei 80\textdegree C lassen
		\item \rarrow Anzeichen, dass es fertig ist: EL rein tunken und wenn die Creme haften bleibt, ist die Konsistenz ungefähr richtig
	\end{description}
	\item Sobald fertig: Flüssigkeit durch eine Sieb in eine Metallschüssel geben und alle Zutaten dazugeben
	\item Metallschüssel über ein Eisbad geben und alles verrühren bis es kalt wird
	\item Getränk in ein Glas geben 
\end{enumerate}