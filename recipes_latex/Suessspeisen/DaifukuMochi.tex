\newpage
\subsection{Daifuku Mochi}
\paragraph{Zutaten}
\begin{itemize}[noitemsep]
	\item 100g Shiratamako 
	\begin{description}[noitemsep, nolistsep]
		\item $\rightarrow$ Alternativ: 115g Klebreismehl
	\end{description}		
	\item 56g Zucker 
	\item 180ml Wasser
	\item 100g Kartoffelstärke/ Speisestärke
	\item 1,5 cup rote Bohnenpaste (Anko)
\end{itemize}
\paragraph{Zubereitung}
\begin{enumerate}[noitemsep]
	\item Klebreismehl und Zucker in eine Schüssel füllen und zusammen verrühren
	\item Wasser hinzugeben und währenddessen rühren
	\begin{enumerate}[noitemsep, nolistsep]
			\item \begin{enumerate}[noitemsep, nolistsep]
						\item Verdampfer-Deckel mit einem Tuch umhüllen
						\item Teig 7,5 Minuten in den Dampfer schmeißen
						\item mit einer nassen Gummi-Spatel gut umrühren
						\item weitere 7,5 Minuten dampfen lassen
					\end{enumerate}
			\item \begin{enumerate}[noitemsep, nolistsep]
						\item Schüssel mit Frischhaltefolie abdecken (nicht zu fest)
						\item 1 Minute lang in die Mikrowelle (1100W)
						\item mit einer nassen Gummi-Spatel umrühren
						\item Schüssel mit Frischhaltefolie abdecken (nicht zu fest)
						\item Mikrowelle: 1100W, 1 Minute 
				\end{enumerate}
	\end{enumerate}
	\begin{description}[noitemsep, nolistsep]
		\item $\rightarrow$ sollte nicht mehr weiß, sondern leicht durchsichtig sein
	\end{description}
	\item ein Backpapier auf die Bearbeitungsfläche ausrollen
	\item Backpapier großzügig mit Kartoffelstärke/ Speisestärke bestäuben
	\item Mochiteig drauf
	\item Mochiteig großzügig mit Kartoffelstärke/ Speisestärke bestäuben
	\item Hände und Nudelholz mit Kartoffelstärke/ Speisestärke vollmachen
	\item Mochi mit Hilfe des Nudelholzes plattwalzen
	\item Backblech unter das Mochi \glqq sliden\grqq{}
	\item 15 Minuten lang im Kühlschrank lassen
	\item Mochiteig herausnehmen und 7-8 Kreise herausschneiden (d=9cm)
		\begin{description}[noitemsep, nolistsep]
			\item $\rightarrow$ mit dem Überbleibseln können später weitere Kreise gemacht werden
		\end{description}
	\item Mochikreis entnehmen
	\item Kartoffelstärke/Speisestärke von den Mochis fegen
	\item auf einen kleinen Teller legen und mit Frischhaltefolie bedecken
	\item kommende Mochis darauf stapeln
	\item Schritt 11-15 wiederholen bis der Teig leer aufgebraucht ist
	\item eine Eisportion Anko nehmen und diese mittig auf einen Mochiteig legen
	\item die vier Seiten des Mochis in die Mitte hin schließen
	\item mit Hilfe der restlichen Ecken die Mochis verschließen
	\item die eben verschlossene Seite in Kartoffelstärke/ Speisestärke tunken 
	\item fertiges Mochi in eine Muffinform legen
\end{enumerate}
\placeimage{Bilder/DaifukuMochi}{11.5cm}{-3cm}{0.6}