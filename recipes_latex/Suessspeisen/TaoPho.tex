\clearpage
\selectlanguage{vietnamese}
\titlelink{Tào Phớ (not done yet)}{https://youtu.be/7f8H8IToruI?si=Jb2LZpEJc1g_Mknb}
\selectlanguage{ngerman}
\label{TaoPho}
\paragraph{Zutaten}
\begin{itemize}[noitemsep]
	\item 350ml Sojamilch 
	\item 20g Zucker (mehr, wenn Sojamilch ungesüßt)
	\item Etwas Ingwer
	\item 1/2 TL Agar Agar
	\item 80ml Wasser(ich glaube, das ist mehr)
\end{itemize}

\paragraph{Zubereitung}
\begin{enumerate}[noitemsep]
	\item Sojamilch in einen Topf geben und Agar Agar auflösen lassen
	\item Sojamilch aufkochen 
	\begin{description}[noitemsep, nolistsep]
		\item \rarrow ggf. Schaum an der Oberfläche ausschöpfen
	\end{description}
	\item Sobald es kocht: Sojamilch auf niedriger Hitze für 30s köcheln lassen
	\item Flüssigkeit nun in eine Schüssel geben
	\item Wasser in einen Topf geben und Zucker darin auflösen lassen
	\begin{description}[noitemsep, nolistsep]
		\item \rarrow Für eine schönere Farbe ggf. Zucker charamelisieren
	\end{description}
	\item Ingwer in Scheiben schneiden und dazugeben
	\item Flüssigkeit für 2 Minuten kochen lassen und anschließend abkühlen lassen
	\item Sobald die Sojamilch-Flüssigkeit fest geworden ist: schichtweise von oben etwas abkratzen und in eine Schüssel geben
	\item Zum Schluss Zuckerwasser hinzugeben
\end{enumerate}