\clearpage
\selectlanguage{vietnamese}
\titlelink{Chè Thập Cẩm (not done yet)}{https://youtu.be/xwq7TLbmqbs?si=rYutyNPJqjFDQqYk}
\selectlanguage{ngerman}
\label{CheThapCam}

\paragraph{Zutaten}
\begin{itemize}[noitemsep]
	\item 350g Kidneybohnen
	\item 100ml Wasser
	\item 50g Zucker
	\item 100g Mungbohnen
	\item 20g Zucker
	\item 5g Vanillezucker (alternativ: 1 TL Vanilleextrakt)
	\item 100ml Wasser (gekocht)
	\item 800ml Kokosnussmilch
	\item 200g Zucker
	\item 10g Tapiokamehl
	\item Weitere Toppings: Litschi, Jackfruit, Erdnuss (geröstet), Eis, Kokosstreifen, Hat Luu, Pandan Jelly,...
\end{itemize}

\paragraph{Zubereitung}
\begin{enumerate}[noitemsep]
	\item Kidneybohnen in einen Sieb geben und das Konservierungswasser unter einem Wasserhahn ausspülen
	\item In enien Topf Zucker und das Wasser geben 
	\item Topf erhitzen und dabei den Zucker darin auflösen lassen
	\item Sobald es kocht: Kidneybohnen hinzufügen und für 5 Minuten köcheln lassen
	\item Mungbohnen kochen anschließend und Zucker, Vanillezucker sowie Wasser mit einem Schneebesen unterrühren
	\item Zucker und Tapiokamehl in einem Topf in der Kokosnussmilch auflösen lassen
	\item Flüssigkeit unter Rühren aufkochen
\end{enumerate}