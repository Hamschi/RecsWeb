\newpage
\titlelink{Panna Cotta (6 Portionen) (not done yet)}{https://youtu.be/XUMGyrLY1TY?si=7KRpmBsSD7Kcsvvf}
\paragraph{Zutaten}
\begin{itemize}[noitemsep]
	\item 375ml Sahne
	\item 100g Zucker
	\item 3 Blatt Gelatine
	\item Wasser (eiskalt)
	\item 40ml Milch
	\item 1/2 Vanilleschote
	\vspace{0.5cm}
	\item 70g Zucker
	\item 200g Beeren (tiefgekühlt)
	\item 1 Sternani
	\item 1 Stange Zimt
	\item Als Garnitur: frische Beeren, Physalis, Minze, etc....
\end{itemize}
\paragraph{Zubereitung}
\begin{enumerate}[noitemsep]
	\item Gelatine in Wasser 5 Minuten einweichen lassen 
	\item Sahne, Milch und Zucker in einen Topf geben
	\item Vanillemark aus der Schote auskratzen und inkl. Schote in den Topf geben
	\item Alles aufkochen
	\item Sobald die Gelatine eingeweicht ist: möglichst viel Wasser aus der Gelatine ausdrücken
	\item Sahne-Milch Mischung vom Herd nehmen und anschließend die Gelatine darin auflösen lassen 
	\item Sahne-Milch Mischung durch einen Sieb in eine Schüssel geben 
	\item Mischung abkühlen lassen, sodass die Flüssigkeit dickflüssig wird
	\item Sobald die Flüssigkeit abgekühlt ist: in Förmchen füllen 
	\item Panna Cotta über Nacht in den Kühlschrank stellen
	\item Zucker in einem Topf karamellisieren (nicht rühren)
	\item Sobald der Zucker goldbraun ist: Die Beeren, den Zimt und die Sternani dazugeben
	\item Alles gut verrühren und eine Minute köcheln lassen
	\item Die Soße pürieren und anschließend durch einen Sieb gießen
	\item Soße auf einem Teller verteilen und anschließend den Panna Cotta aus der Form nehmen und darauf legen
	\item Tipp: Im Video ist noch eine Anleitung für Zuckerdeko
	\item Tipp: Panna Cotta 20 Minuten vor dem Servieren aus dem Kühlschrank nehmen
\end{enumerate}
