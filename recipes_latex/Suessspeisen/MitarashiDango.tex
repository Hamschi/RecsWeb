\newpage
\titlelink{Mitarashi Dango (not done yet)}{https://youtu.be/CH_u8Pjw5oM?si=YpryvBzk5uSS8cYO}
\paragraph{Zutaten}
\begin{itemize}[noitemsep]
	\item 200g Klebreismehl
	\item 1 Stk. Tofu (300g)
	\item 4 EL Zucker
	\item 4 EL Sojasoße
	\item 200ml Wasser
	\item 1 EL Kartoffelstärke
\end{itemize}
\paragraph{Zubereitung}
\begin{enumerate}[noitemsep]
	\item Tofu zerquetschen und zu nem Brei verarbeiten
	\item Klebreismehl und den Tofu miteinander verkneten bis es so weich wie ein Ohrläppchen(?) wird
	\item Teig in Kugeln formen und auf Backpapier o.Ä. auslegen
	\begin{description}[noitemsep, nolistsep]
		\item \rarrow Durchmesser 3-4cm
		\item \rarrow werden evtl. schöner, wenn man nasse Hände nutzt
	\end{description}
	\item Dangos in einem Topf kochen
	\item Sobald die Dangos an der Oberfläche schwimmen: in einen Behälter mit kalten/lauwarmen Wasser geben
	\item Dangos anschließend mit Bambusspießen spießen
	\begin{description}[noitemsep, nolistsep]
		\item \rarrow Spieße vorher in Wasser tauchen, um sie leichter spießen zu lassen
	\end{description}
	\item Dango-Spieße in einer Pfanne mit Öl anbraten, um denen Farbe zu verleihen
	\item Für die Soße Zucker und Sojasoße in eine Pfanne/ einen Topf geben und den Zucker darin auflösen lassen (bei Zufuhr von Hitze)
	\item Anschließend Wasser dazugeben und vermengen
	\item Kartoffelstärke darin auflösen lassen
	\item Soße kochen lassen bis die Soße dickflüssig wird
	\item Soße in einen hohen Behälter geben und die Dangos darin eintauchen lassen
\end{enumerate}