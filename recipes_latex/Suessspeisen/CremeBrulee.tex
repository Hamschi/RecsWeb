\newpage
\subsection{Crème brûlée}
\paragraph{Zutaten}
\begin{itemize}[noitemsep]
	\item 250ml Vollmilch
	\item 250ml Sahne (35\% Fett)
	\item 1 EL Vanille-Extrakt 
	\item 4 Eigelbe (groß)
	\item 90g Zucker (weiß)
	\item 50g Zucker (braun)
\end{itemize}
\paragraph{Zubereitung}
\begin{enumerate}[noitemsep]
	\item Heize den Ofen auf 100\textdegree C vor
	\item Milch und Sahne in einem Topf bei niedriger Hitze vermengen
	\item Füge den Vanille-Extrakt hinzu 
	\item Bringe die Flüssigkeit zum Kochen, während du das kontinuierlich umrührst
	\item Sobald es brodelt: Topf sofort vom Herd nehmen und Herd ausschalten
	\item Vermenge die Eigelbe mit dem Zucker
	\begin{description}[noitemsep, nolistsep]
		\item $\rightarrow$ Nutze einen Kochlöffel aus Holz und keinen Schneebesen, weil wir keinen Schaum haben wollen
	\end{description}
	\item Sobald die Flüssigkeit abgekühlt ist: Gebe einen Drittel der Flüssigkeit durch einen Sieb zu der Zucker-Ei-Masse 
	\item Verrühre alles sanft bis der Zucker sich aufgelöst hat
	\item Gebe nun den Rest der Flüssigkeit durch den Sieb in die Zucker-Ei-Masse und vermenge alles wieder mit dem Kochlöffel
	\item Lege deine Auflaufformen nun auf ein mit Backpapier oder Alufolie ausgelegtem Blech
	\item Kippe gleichmäßig viel Flüssigkeit durch den Sieb in die einzelnen Auflaufformen
	\item Backe die Cremes nun für 45-60 Minuten im Ofen bei 100\textdegree C
	\begin{description}[noitemsep, nolistsep]
		\item $\rightarrow$ Der Teig ist fertig, wenn die Kanten 	fest und die Mitte wabbelig ist
		\item Tap Test: Tippe leicht an die Seite deiner Backformen. Wenn nur ein leichtes Wabbeln zu sehen ist, sind die fertig
	\end{description}
	\item Sobald der Tap Test überstanden ist: Nehme die Backformen aus dem Ofen und lasse sie für 30-40 Minuten kühlen
	\item Lege sie anschließend in den Kühlschrank für mindestens 2h (am besten über Nacht)
	\item Bevor die Cremes serviert werden, verteile möglichst gleichmäßig braunen Zucker darüber
	\item Karamellisiere den Zucker nun mit einem Flambierbrenner, durch Flambieren mit Rum oder im Ofen bei hoher Hitze
\end{enumerate}
\placeimage{Bilder/CremeBrulee}{11.5cm}{-4.5cm}{0.5}