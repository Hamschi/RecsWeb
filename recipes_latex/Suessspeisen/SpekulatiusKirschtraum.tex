\newpage
\subsection{Spekulatius-Kirschtraum (2 Portionen)}
\paragraph{Zutaten}
\begin{itemize}[noitemsep]
	\item 250g Sauerkirschen (tiefgefroren)
	\item 2 EL Zucker
	\item 1 TL Zimt
	\item 1,5 EL Stärke (gehäuft)
	\item 6 Spekulatius Kekse
	\item 125g Mascarpone
	\item 125g Frischkäse
	\item 70g Sahne
	\item 70g Crème fraîche
	\item 1 Orangenschale
	\item 1 Vanilleschote
	\item 2 EL Puderzucker (gehäuft)
	\item Kakao, Zimt, Zucker und Zitronenschale zum Bestreuen
\end{itemize}
\paragraph{Zubereitung}
\begin{enumerate}[noitemsep]
	\item die Kirschen in einem Topf auftauchen lassen 
	\item Danach: Zucker und Zimt hinzufügen und verrühren
	\item Stärke mit etwas Wasser anrühren
	\item vereinzelt wenige Milliliter in die Kirschen geben und dabei umrühren bis keine Stärke mehr übrig ist 
	\item Danach: die Kirschen abkühlen lassen 
	\begin{description}[noitemsep, nolistsep]
		\item $\rightarrow$ Henssler hat sie extra in eine andere Pfanne gekippt
	\end{description}
	\item die Vanilleschote auskratzen
	\item in einer Schüssel Crème fraîche, Mascarpone, Sahne, Frischkäse, Puderzucker ausgekratzte Vanilleschote sammeln
	\item Orangenschale in die Schüssel reiben und alles miteinander verrühren
	\item den Spekulatius in kleine Bröckchen stampfen (nicht zu klein!) 
	\item mit dem Kirschkompott im Glas einen Boden bilden
	\item die Kirschen mit den Keksen bedecken
	\item das Glas mit der Creme bis zu 6/7 füllen
	\item die Creme mit Kakaopulver, Zimt und Zucker bestreuen
\end{enumerate}
\placeimage{Bilder/SpekulatiusKirschtraum}{14.5cm}{-4cm}{0.35}