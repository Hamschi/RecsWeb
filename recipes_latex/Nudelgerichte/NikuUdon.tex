\newpage
\titlelink{Niku Udon (2 Portionen)}{https://www.youtube.com/watch?v=qZod329yWAo&list=PLwXqhx9ZMa_VsqCvTGJoXphf10mSI_Jf&index=166&ab_channel=Sudachi\%7CJapaneseRecipes}

\paragraph{Zutaten}
\begin{itemize}[noitemsep]
	\item 150g Rindfleisch (dünn geschnitten)
	\item 1 EL Sojasoße
	\item 1 EL Mirin
	\item 1 TL Austernsoße
	\item 1/2 TL Zucker (braun)
	\item 1/2 TL Ingwer (gerieben)
	\item 50ml Wasser
	\item 1/2 Zwiebel 
	\item 1/2 TL Salz
	\vspace{0.5cm}
	\item 500ml Dashi Brühe (Instant geht auch)
	\item 1 EL Sojasoße
	\item 1 EL Mirin
	\item 1 TL Zucker (braun)
	\vspace{0.5cm}
	\item 1 TL Öl
	\item 2 Portionen Udon Nudeln
	\item Frühlingszwiebeln 
	\item Japanisches Chilipulver (shichimi togarashi, optional)
\end{itemize}

\paragraph{Zubereitung}
\begin{enumerate}[noitemsep]
	\item Einen halbwegs großen Behälter nehmen: Sojasoße, Mirin, Austernsoße, Zucker (braun), Ingwer und Wasser zu einer Marinade verrühren
	\item Sobald die festen Zutaten sich aufgelöst haben: Fleisch reingeben und 10 Minuten marinieren
	\item Zwiebel in Wedges-Scheiben schneiden 
	\item Zwiebeln in eine Schüssel geben und mit Salz bestreuen und alles verrühren
	\item Dashibrühe, Sojasoße, Mirin, Zucker (braun) in einem Topf erhitzen und 2 Minuten kochen lassen
	\item Öl erhitzen und Zwiebeln glasig anbraten
	\item Danach: Fleisch inkl. Marinade hinzugeben und braten, bis das Fleisch nicht mehr rot ist
	\item Nudeln in zwei verschiedene Schüsseln aufteilen und die Brühe gleichmäßig hinzugeben
	\item Fleisch mit Zwiebeln darüber geben, inkl. bisschen Brat-Flüssigkeit
	\item Mit Frühlingszwiebeln und jap. Chili Pulver garnieren (optional)
\end{enumerate}
\placeimage{Bilder/NikuUdon}{11.5cm}{-4cm}{0.6}