\newpage
\subsection{Spaghetti Pomodore}
\paragraph{Zutaten}
\begin{itemize}[noitemsep]
	\item 300g Spaghetti
	\item 20 mittelgroße Rispentomaten
	\item 1 Knoblauchzehe
	\item Salz und Pfeffer
	\item 1/2 Bund Basilikum
	\item 1 Butterflocke
	\item Olivenöl
\end{itemize}
\paragraph{Zubereitung}
\begin{enumerate}[noitemsep]
	\item Tomaten in eine Schüssel Wasser legen
	\item Nudeln anfangen al dente zu kochen 
	\item Olivenöl in einer Pfanne bei mittlerer Hitze erwärmen 
	\begin{description}[noitemsep, nolistsep]
		\item $\rightarrow$ Knoblauch soll darin nur leicht simmern 
	\end{description}
	\item Knoblauch mit einem großen Messer zerdrücken und ins Olivenöl legen 
	\item die Tomaten in der Schüssel unter Wasser zerdrücken, leicht zerreiben und dann einzeln in die Pfanne geben
	\item mit Salz und Pfeffer würzen 
	\item Zwischendurch: eine Kelle Kochwasser in die Pfanne geben und die Hitze erhöhen 
	\item Sobald die Nudeln al dente sind: zu der Pfanne geben und vermengen
	\item Basilikumblätter zu den Nudeln geben
	\item Butter in die Nudeln einrühren und schmelzen lassen
\end{enumerate}
\vspace{1cm}
\begin{figure}[h]
\centering
\includegraphics[width=.9\textwidth]{Bilder/Pomodore}
\end{figure}