\newpage
\selectlanguage{vietnamese}
\subsection{Mì xào giòn thịt bò}
\selectlanguage{ngerman}
\paragraph{Zutaten}
\begin{itemize}[noitemsep]
	\item 2 Pck. Instantnudeln
	\item Öl zum Bestreichen vom Backblech
	\item 400g Rindfleisch
	\item Pfeffer
	\item Knoblauchpulver (so viel wie Pfeffer)
	\item 1 Ei
	\item 1 Schalenboden (mittelgroß) Wasser
	\item 1 Prise Baking Soda 
	\item 2 EL Speisestärke
	\item 1 EL Öl
	\vspace{0.5cm}
	\item 150g Sojasprossen
	\item 1 Schalenboden (mittelgroß) Wasser
	\item 1 TL Speisestärke (gehäuft)
	\item 1/2 TL Salz
	\item 1,5 TL Hähnchenbrühe
	\item 2 TL Zucker
	\item 1/2 TL dunkle Sojasoße
	\item 1 TL helle Sojasoße
	\item 1,5 EL Austersoße
	\item 8 EL Wasser
	\item Öl zum Braten
	\item 1 Zehe Knoblauch
\end{itemize}
\paragraph{Zubereitung}
\begin{enumerate}[noitemsep]
	\item Nudeln 2-3 Minuten in kochendem Wasser kochen lassen
	\item Öl gleichmäßig auf das Backblech verteilen und in den Backofen tun
	\item Rindfleisch mit Tuch abtropfen und in längliche Streifen schneiden
	\item Pfeffer, Knoblauchpulver und ein Ei zum Rindfleisch geben
	\item Baking Soda im Wasser auflösen und zum Fleisch geben
	\item alles mischen
	\item Maisstärke zum Fleisch geben und Fleisch verkneten
	\begin{description}[noitemsep, nolistsep]
		\item $\rightarrow$ Fleisch darf nicht wässrig sein
	\end{description}
	\item Öl zum Fleisch geben und Fleisch 30 Minuten ziehen lassen
	\vspace{0.5cm}
	\item Ofen auf 200\textdegree C vorheizen
	\item Nudeln 6 Minuten in den Ofen legen
	\item Maisstärke in dem Wasser auflösen und zur Seite legen
	\item Salz, Brühe, Zucker, Sojasoßen, Wasser verrühren und zur Seite legen
	\item Öl auf einem Herd (Herdstufe 8) erhitzen
	\item Sojasprossen darin 1-2 Minuten lang anbraten
	\item Öl auf einem Herd erhitzen
	\item Fleisch darin 1 Minuten anbraten (es ist noch rot)
	\item Öl auf einem Herd erhitzen
	\item Knoblauch kleinschneiden und darin anbraten
	\item Fleisch eine weitere Minute da rein
	\item Sojasprossen eine Minute rein
	\item Soße (Mischung) reinkippen
	\item Wasser mit der Maisstärke reinkippen
	\item Fleisch auf die Nudeln schütten
\end{enumerate}
\vspace{3cm}
\begin{figure}[h]
\centering
\includegraphics[width=.9\textwidth]{Bilder/RamenKnusp}
\end{figure}