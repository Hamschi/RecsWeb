\newpage
\subsection{Lasagna (4 Portionen)}
\paragraph{Zutaten}
\begin{itemize}[noitemsep]
	\item 100g Butter 
	\item 80g Mehl
	\item 1l Milch
	\item 100g Parmesan
	\item 1 Prise Salz
	\item 1 Prise Pfeffer 
	\item 1/4 Muskatnuss
	\item 500g Lasagneblätter  
	\item Ragu (Bolognese?)
	\item 1 Mozzarella 
	\item noch mehr Parmesan 
\end{itemize}
\paragraph{Zubereitung}
\begin{enumerate}[noitemsep]
	\item Butter in einem Topf schmelzen 
	\item Herd auf mittlere Hitze stellen und Mehl dazugeben und verrühren 
	\item Milch dazugeben und verrühren
	\item Den Parmesan darin auflösen lassen 
	\item Salz, Pfeffer und Muskatnuss hinzugeben und alles 10-12 Minuten lang verrühren bis die Soße dickflüssig wird 
	\begin{description}[noitemsep, nolistsep]
		\item $\rightarrow$ Es sollte unten nichts anbrennen 
	\end{description}
	\item Bisschen Ragu auf die Auflaufform verteilen 
	\item Eine Schicht Lasagneblätter in die Form geben (müssen wahrscheinlich geschnitten werden
	\item Ragu über diese Blätter geben 
	\item Die Weiße Soße darüber geben 
	\item Mozzarella zerreißen und darüber geben 
	\item Parmesan darüber streuen 
	\item Schritte 7-11 wiederholen bis die Auflaufform gefüllt ist
	\item Auflaufform mit Alufolie bedecken und für 35-45 Minuten in den Backofen geben bei 180\textdegree C
	\item Entferne für die letzten 5 Minuten die Alufolie von der Auflaufform 
\end{enumerate}
\placeimage{Bilder/Lasagna}{11.5cm}{-4.5cm}{0.6}