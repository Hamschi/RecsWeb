\newpage
\subsection{Pasta Cacio E Pepe (2 Portionen)}
\paragraph{Zutaten}
\begin{itemize}[noitemsep]
	\item 300g Spaghettoni 
	\item 200g Pecorino Romano
	\item Pfeffer (aus der Mühle)
	\item Steinsalz
	\item 3l Wasser 
\end{itemize}
\paragraph{Zubereitung}
\begin{enumerate}[noitemsep]
	\item Wasser aufkochen und dann das Salz dort reingeben
	\item die Nudeln darin al dente kochen
	\item Pfeffer für paar Minuten großzügig und gleichmäßig auf die Oberfläche der warmen Pfanne mahlen (mittlere Hitze) 
	\item Sobald der Pfeffergeruch kräftig ist: eine 2/3 Suppenkelle Nudelwasser zu dem Pfeffer geben
	\item Käse in eine Schale reiben und 1/4 Suppenkelle Nudelwasser in die Schale geben und zu einer knetigen Masse formen 
	\item Nudeln zu dem Pfeffer geben
	\item eine Suppenkelle Nudelwasser zu den Nudelngeben und gleichmäßig umrühren
	\begin{description}[noitemsep, nolistsep]
		\item $\rightarrow$ ggf. mehr Nudelwasser dazugeben, falls zu trocken (Nudeln müssen darin weiterkochen können
	\end{description}
	\item mehr Pfeffer über die Nudeln streuen
	\item Nudeln darin fertigkochen
	\item den Herd ausschalten und den Käse in die Pfanne geben
	\item alles zu einer cremigen Soße vermengen
	\begin{description}[nolistsep, noitemsep]
		\item $\rightarrow$ Käsefäden meiden
		\item $\rightarrow$ Pan-Tosses (oder so) hat der Dude zumindest gemacht
	\end{description}
	\item Nudeln mit Käse und Pfeffer on top servieren
\end{enumerate}
\begin{figure}[h]
\centering
\includegraphics[width=0.75\textwidth]{Bilder/CacioEPepe}
\end{figure}
