\newpage
\subsection{Pad Thai (2 Portionen) (not done yet)}
\paragraph{Zutaten}
\begin{itemize}[noitemsep]
	\item 3 EL Palm Sugar
	\item 3 EL Wasser
	\item 3-4 EL Tamarind Paste
	\item 2 EL Fischsoße
	\vspace{0.5cm}
	\item 115g Reisnudeln
	\item 2 EL Garnelen (getrocknet, gehackt)
	\item 3 Zehen Knoblauch (gehackt)
	\item 1 Schalotte (gehackt)
	\item 85g Tofu (gepresst, gewürfelt)
	\item 3 EL Daikon Radish (gehackt)
	\item Chiliflocken
	\item 3 EL Pflanzenöl
	\item 10 Garnelen (medium-size)
	\item 2 Eier
	\item 120g Sojasprossen
	\item 7-10 Stück Knoblauch-Schnittlauch 
	\item 55g Erdnüsse (geröstet, gehackt)
	\item 1 Limette
\end{itemize}

\paragraph{Zubereitung}
\begin{enumerate}[noitemsep]
	\item Palm Sugar in einem Topf zum schmelzen bringen und eine dunkle Farbe annehmen lassen
	\item Danach: Wasser, Fischsoße und Tamarind Paste hinzufügen
	\item Reisnudeln für eine Stunde in lauwarmes Wasser legen 
	\item Reisnudeln anschließend halbieren
	\item Garnelen (getrocknet), Schalotten, Knoblauch, Daikon Radish und Chiliflocken zum Tofu hinzugeben
	\item Enden der Knoblauch-Schnittlauche wegschneiden und den Schnittlauch in ca. 7cm Stücke schneiden
	\item Schnittlauch zu den Sojasprossen geben 
	\item Die Garnelen beidseitig in Öl anbraten und anschließend zur Seite legen
	\item In der gleichen Pfanne Öl nachgeben und den Tofu darin anbraten
	\item Sobald die Schalotten goldbraun sind: Reisnudeln und die Soße hinzufügen und solange braten, bis die Nudeln die gesamte Soße aufgesogen haben 
	\item Platz in der Pfanne machen und die Eier reinlegen
	\item Das Eigelb zerteilen und die Nudeln AUF das Ei legen und für ca. 30s braten
	\item Die Hälfte der Erdnüsse und des Gemüsen reingeben und durchbraten
	\item Anschließend mit einer Limettenscheibe, Erdnüssen, Sojasprossen, Schnittlauch und den Garnelen verzieren
\end{enumerate}