\newpage
\subsection{Ramen}
\paragraph{Zutaten}
\begin{itemize}[noitemsep]
	\item Toppings, z.B. Aji Tamago, Chashu, Menma, Frühlingszwiebeln, Nori Blätter, Spinat, Naruto Cake, Narutomaki,...
	\item 1 kl. Suppenkelle (1.5 Unzen) Tare
	\begin{description}[noitemsep, nolistsep]
		\item $\rightarrow$ Shoyu Tare für Shoyu Ramen
		\item $\rightarrow$ Shio Tare für Shio Ramen 
	\end{description}
	\item 1/2 TL Green Onion Aroma Öl 
	\item 1/2 TL Bonito Aroma Öl 
	\item 1 gr. Suppenkelle (12 Unzen) Chintan Suppe 
	\item 1 Portion Nudeln (nicht feste Nudeln)
\end{itemize}
\paragraph{Zubereitung}
\begin{enumerate}[noitemsep]
	\item Chintan Suppe erhitzen
	\item Shoyu Tare und die Aroma Öle in die Schüssel geben
	\item Nudeln auf einem Sieb kochen und mit Stäbchen dafür sorgen, dass sie nicht mehr zusammen kleben 
	\begin{description}[noitemsep, nolistsep]
		\item $\rightarrow$ sollte nur 90s in Anspruch nehmen
	\end{description}
	\item 30s Nudeln weiter im Wasser lassen
	\item Währenddessen: Chintan Suppe in die Schüssel geben 
	\item Danach: Wasser von den Nudeln \glqq{}schlagen\grqq{} 
	\item Nudeln am Rand der Schüssel vom Sieb in die Schüssel rutschen lassen  
	\begin{description}[noitemsep, nolistsep]
		\item $\rightarrow$ Versuchen eine Insel zu bilden für die Toppings 
	\end{description}
	\item Toppings drauf
\end{enumerate}
\placegraphic{Bilder/Shoyu_Ramen}{0.85}