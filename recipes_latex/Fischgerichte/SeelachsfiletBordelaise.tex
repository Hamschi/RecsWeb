\newpage
\subsection{Seelachsfilet à la Bordelaise (not done yet)}
\paragraph{Zutaten}
\begin{itemize}[noitemsep]
	\item 400g Seelachsfilet
	\item Salz und Pfeffer
	\item 1 Zwiebel (klein)
	\item 1 Zehe Knoblauch
	\item 1 EL Olivenöl
	\item 20g Butter
	\item 30g Semmelbrösel (oder Paniermehl)
	\item 1 EL Thymian (gerebelt)
	\item 3 EL Petersilie (gehackt)
	\item 1 Tasse Fleischbrühe 
	\item 1/2 Zitronen Zitronensaft
	\item Butter für die Form
\end{itemize}
\paragraph{Zubereitung}
\begin{enumerate}[noitemsep]
	\item den Fisch waschen, trocken tupfen und anschließend salzen und pfeffern
	\item Die Zwiebel und den Knoblauch feinschneiden und in Olivenöl mit Butter glasig dünsten
	\item Semmelbrösel und Kräuter mit den Zutaten in der Pfanne vermengen, salzen und pfeffern
	\begin{description}[noitemsep, nolistsep]
		\item $\rightarrow$ Alternative zu den Kräutern: 4 EL TK-8-Kräutermischung
	\end{description}
	\item Backofen auf 180\textdegree C Ober-/Unterhitze vorheizen
	\item Auflaufform mit Butter beschmieren und den Fisch dort reinlegen
	\item Die Bröselmasse auf den Fisch verteilen 
	\item etwas von der heißen Brühe über den Fisch gießen
	\begin{description}[noitemsep, nolistsep]
		\item $\rightarrow$ oder meinte er in die Auflaufform?
		\item $\rightarrow$ Link: \url{https://www.chefkoch.de/rezepte/1175301223643082/Schlemmerfilet-Bordelaise.html}
	\end{description}
	\item 5 Minuten im Backofen backen und dann nachschauen, ob die Brühe verkocht ist
	\begin{description}[noitemsep, nolistsep]
		\item $\rightarrow$ Falls ja: Etwas Brühe nachfüllen
	\end{description}
	\item weitere 5-7 Minuten backen
	\item den Fisch aus dem Ofen holen und am Tisch mit Zitronensaft begießen
\end{enumerate}