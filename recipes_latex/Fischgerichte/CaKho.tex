\titlelink{}{}
\paragraph{Zutaten}
\begin{itemize}[noitemsep]
	\item Zuckerwasser (braun) (alternativ: dunkle Sojasoße) für die Farbe, Verhältnis selber schätzen
\end{itemize}

\paragraph{Zubereitung}
\begin{enumerate}[noitemsep]
	\item Fisch in Salzwasser waschen 
	\item Fisch in Tee waschen 
	\item Die Köpfe der Frühlingszwiebeln abschneiden und zur Seite legen und den grünen Teil in halbwegs dünne Rohre schneiden
	\item Zwiebeln und Knoblauch grob hacken
	\item Öl in den heißen Topf geben und sobald es heiß wird die Zwiebeln und den Knoblauch dazu geben und kurz anbraten
	\item Zwiebeln und Knoblauch aus dem Öl nehmen
	\item Fischsoße, Zucker, Hühnerbrühepulver, MSG, Kokoswasser, Wasser, Sojasoße, Pfeffer, Chilisoße in das Öl geben
	\item Flüssigkeit zum Kochen bringen
	\item Währenddessen: Fisch für 3 Minuten in kalten Tee geben
	\item Sobald die Flüssigkeit kocht: vom Herd nehmen und in einen anderen Behälter kippen
	\item Zwiebeln und Knoblauch fein hacken
	\item Zwiebeln und Knoblauch anbraten und darin dann die Seiten des Fischs anbraten, damit der Fisch besser zusammenhält
	\item Fertigen Fisch in einem Topf sammeln
	\item Die Kochflüssigkeit kurz anrühren und davon so viel zum Fisch geben bis sie bedeckt werden
	\item Knoblauchzehen andrücken und dazugeben
	\item Fisch nun für 10 Minuten bei niedriger Hitze köcheln lassen, Deckel auf dem Topf lassen
	\item Danach: Fisch wenden, Frühlingszwiebeln dazugeben und ohne Deckel köcheln lassen
\end{enumerate}